%%%%%%%%%%%%%%%%%%%%%%%%%%%%%%%%%%%%%%%%
%                                      %
% Random LaTeX Beamer Template         %
%                                      %
% Created by Thomas Dunmore.           %
% Last updated 2024/11/07.             %
%                                      %
%%%%%%%%%%%%%%%%%%%%%%%%%%%%%%%%%%%%%%%%

%%%%%%%%%
% SETUP %
%%%%%%%%%

\documentclass{beamer}
\mode<presentation> {
	\usetheme{Madrid}
}
\setbeamertemplate{frametitle continuation}[from second][]

\usepackage{csquotes}
\usepackage{graphicx}
\usepackage{booktabs}

% Title.

\title[Ribbon Categories and RT Invariants]{Ribbon Categories and Reshetikhin-Turaev Invariants}%\title[MTCs and RT Invariants]{Modular Tensor Categories and Reshetikhin-Turaev Invariants}
\author[Thomas Dunmore]{Thomas Dunmore}
\institute[UNSW]{
	University of New South Wales \\
	\medskip
	\textit{t.dunmore@unsw.edu.au}
}
\date{2025/06/16}

% Bibliography.
\usepackage[%
	backend = biber,
	% BibLaTeX-Math package: https://github.com/konn/biblatex-math
	style = math-alphabetic,
	giveninits = true,
	dashed = false,
	url = false,
	doi = false,
	sorting = anyvt,
	minalphanames = 3,
	maxalphanames = 4,
	maxcitenames = 5,
	maxbibnames = 5
]{biblatex}
% Use the default font size for the bibliography.
\renewcommand*{\bibfont}{\normalsize}
% Use title case rather than sentence case for references.
\DeclareFieldFormat{titlecase}{#1}
% Put last names before first names.
\DeclareNameAlias{default}{family-given}
% Used for articles with appendices written by other authors.
\NewBibliographyString{bywithappendix}
\DefineBibliographyStrings{english}{
	bywithappendix = {with an appendix by}
}
% Specify the bibliography data file to use.
\addbibresource{references.bib}
% Automatically add every reference to the bibliography.
\nocite{*}

% Plotting.
\usepackage{tikz-cd}
\usepackage{spath3}
\usetikzlibrary{arrows.meta, decorations.markings, decorations.pathreplacing, bending, knots}


%%%%%%%%%%%%
% NOTATION %
%%%%%%%%%%%%

\usepackage{mathrsfs}
% Reduce spacing between script letters.
\newcommand{\multiscr}[1]{\text{\usefont{U}{rsfs}{m}{n}#1}}

% Adds integral notation like \oiint.
\usepackage{esint}
% Blackboard bold symbols.
\usepackage{bbm}

\usepackage{accents}
% Tilde notation for vectors.
\newcommand{\ut}[1]{\underaccent{\tilde}{#1}}
% Arrow notation for vectors.
\usepackage{harpoon}
% Dirac bra-ket notation for quantum states.
\usepackage{braket}

% Differential formatting.
\usepackage{ifthen}
\usepackage{etoolbox}
\newcommand*{\ndiff}[1]{\mathrm{d}#1}
\newcommand*{\sdiff}[1]{\mathop{}\!\ndiff{#1}}
\newcommand{\rdiff}[3][]{
	\ifthenelse{\equal{#1}{}}
	{\frac{\mathrm{d}#2}{\mathrm{d}#3}}
	{\frac{\mathrm{d}^{#1}#2}{\forcsvlist\ndiff{#3}}}
}
\newcommand*{\npiff}[1]{\mathrm{\partial}#1}
\newcommand*{\spiff}[1]{\mathop{}\!\npiff{#1}}
\newcommand{\rpiff}[3][]{
	\ifthenelse{\equal{#1}{}}
	{\frac{\mathrm{\partial}#2}{\mathrm{\partial}#3}}
	{\frac{\mathrm{\partial}^{#1}#2}{\forcsvlist\npiff{#3}}}
}
% Inexact differential for physics.
\newcommand*{\dbar}[1]{\mathop{}\!\mathrm{\dj}#1}

% Metrics, inner products and norms.
\usepackage{mathtools}
\DeclarePairedDelimiter{\abs}{\lvert}{\rvert}
\DeclarePairedDelimiter{\inprod}{\langle}{\rangle}
\DeclarePairedDelimiter{\norm}{\lVert}{\rVert}
% This is used if we want an empty norm. 
\newcommand{\blank}{{}\cdot{}}

% Function notation.
\newcommand{\id}{\textup{id}}
\newcommand{\coker}{\textup{coker}}
\newcommand{\im}{\textup{im}}
\newcommand{\ev}{\textup{ev}}
\newcommand{\coev}{\textup{coev}}

% Category font.
\newcommand{\mathcat}[1]{\mathcal{#1}}

% Category theory notation.
\newcommand{\Ob}{\textup{Ob}}
\newcommand{\Hom}[2][]{
	\ifthenelse{\equal{#2}{}}
	{\textup{Hom}_{#1}}
	{\textup{Hom}_{#1}\!\left(#2\right)}
}
\newcommand{\IntHom}[2][]{
	\ifthenelse{\equal{#2}{}}
	{\underline{\textup{Hom}}_{#1}}
	{\underline{\textup{Hom}}_{#1}\!\left(#2\right)}
}
\newcommand{\End}[2][]{
	\ifthenelse{\equal{#2}{}}
	{\textup{End}_{#1}}
	{\textup{End}_{#1}\!\left(#2\right)}
}
\newcommand{\IntEnd}[2][]{
	\ifthenelse{\equal{#2}{}}
		{\underline{\textup{End}}_{#1}}
		{\underline{\textup{End}}_{#1}\!\left(#2\right)}
}
\newcommand{\Fun}[2][]{
	\ifthenelse{\equal{#2}{}}
	{\textcat{Fun}_{#1}}
	{\textcat{Fun}_{#1}\!\left(#2\right)}
}
\newcommand{\opcat}[1]{{#1}^{\textup{op}}}
\newcommand{\revcat}[1]{{#1}^{\textup{rev}}}
\newcommand{\textcat}[1]{\textup{\textsf{#1}}}
\newcommand{\rmodcat}[2][]{\textcat{Mod}_{#1}\textcat{-}#2}
\newcommand{\lmodcat}[2][]{#2\textcat{-Mod}_{#1}}
\newcommand{\bimodcat}[3][]{#2\textcat{-Mod}_{#1}\textcat{-}#3}

% Special notation.
\newcommand{\chr}{\textup{char}}
\newcommand{\Tr}{\textup{Tr}}
\newcommand{\trv}{\textup{tr}}
\newcommand{\Irr}{\textup{Irr}}
\newcommand{\Gr}{\textup{Gr}}
\newcommand{\dimh}[2]{\left(#1, #2\right)}
\newcommand{\Dim}{\textup{Dim}}
\newcommand{\FPdim}{\textup{FPdim}}
\newcommand{\Wr}{\textup{Wr}}
\newcommand{\BrPic}[1][]{
	\ifthenelse{\equal{#1}{}}
	{\textup{BrPic}}
	{\textup{BrPic}\!\left(#1\right)}
}
\newcommand{\BrPicC}[1][]{
	\ifthenelse{\equal{#1}{}}
	{\underline{\BrPic}}
	{\underline{\BrPic}\!\left(#1\right)}
}
\newcommand{\BrPicCC}[1][]{
	\ifthenelse{\equal{#1}{}}
	{\underline{\BrPicC}}
	{\underline{\BrPicC}\!\left(#1\right)}
}
\newcommand{\Out}{\textup{Out}}
\newcommand{\OutC}{\underline{\Out}}
\newcommand{\OutCC}{\underline{\OutC}}

% Hiragana "yo" for the Yoneda embeddings.
\newcommand{\yo}{\text{\usefont{U}{min}{m}{n}\symbol{'210}}}
\DeclareFontFamily{U}{min}{}
\DeclareFontShape{U}{min}{m}{n}{<-> udmj30}{}

% Representation theory notation.
\newcommand{\Sym}{\textup{Sym}}
\newcommand{\Alt}{\textup{Alt}}


%%%%%%%%%%%%
% DOCUMENT %
%%%%%%%%%%%%

\begin{document}

%%%%%%%%%%%%
% Overview %
%%%%%%%%%%%%

\begin{frame}
\titlepage
\noindent\\[-15pt]
\begin{figure}[!ht]
\hspace{3.5cm}\hfill\includegraphics[width=2cm]{unsw-crest}\hfill\raisebox{-0.7093cm}{\includegraphics[width=3.5cm]{qr-code}}
\end{figure}
\end{frame}

%\begin{frame}
%\frametitle{Overview}
%\begin{center}
%\begin{minipage}{\widthof{(1) Reshetikhin-Turaev Invariants}}
%\setlength{\parskip}{4ex}
%\tableofcontents
%\end{minipage}
%\end{center}
%\end{frame}

%%%%%%%%%%%%%%%%%%%%%%%%%%%%%%%%%
% Reshetikhin-Turaev Invariants %
%%%%%%%%%%%%%%%%%%%%%%%%%%%%%%%%%

\section{Reshetikhin-Turaev Invariants}

%\begin{frame}
%\centerline{\Huge\textcolor{structure}{\underline{Reshetikhin-Turaev Invariants}}}
%\end{frame}

\begin{frame}
\frametitle{A Brief History}
In 1984, Vaughan Jones stumbled upon a new polynomial invariant for knots. Somewhat unsatisfyingly, all known definitions were intrinsically 2-dimensional, despite links naturally being 3-dimensional objects.
\newline\newline
In 1989, Witten solved this ``problem'' by constructing the Jones polynomial (and other invariants of links and 3-manifolds) from special kinds of 2 + 1d topological quantum field theories.
\newline\newline
The following year, in 1990, Reshetikhin and Turaev gave a mathematical formulation of Witten's construction in terms of ribbon graphs coloured by representations of quantum groups.
\end{frame}

\begin{frame}
\frametitle{Categories for the Working Mathematician}
Remember from Victor's talk that a category is (loosely) a collection of objects along with a collection of arrows (morphisms) between them.
\newline\newline
Our running example will be \textcolor{structure}{$\textcat{Vec}$}, the category whose objects are \textcolor{structure}{finite-dimensional} vector spaces (say, over $\mathbb{C}$), and whose morphisms are linear maps (morphisms $\mathbb{C}^m \to \mathbb{C}^n$ are $n \times m$ complex matrices).
\newline\newline
Note that $\textcat{Vec}$ has some nice properties that will be important to us.
\begin{itemize}
	\item It has finitely many simple objects, and every object is a finite direct sum of them. In $\textcat{Vec}$, only the one-dimensional vector space $\mathbb{C}$ has no non-trivial subspaces.
	\item The morphisms between any two objects form a finite-dimensional vector space.
\end{itemize}
We will ask that all of our categories satisfy these properties.
\end{frame}

\begin{frame}
\frametitle{The Most Useless Graphical Calculus Ever}
In any category, we can represent morphisms using diagrams. Great!
\begin{center}
\begin{tikzpicture}[baseline = (current bounding box.east)]
\draw[thick, arrows = {-Latex}] (0, 1.5) to (0, 0);
\node at (0, 0.75) [left] {$\id_X =\ $};
\node at (0, 1.5) [above] {$X$};
\node at (0, 0) [below] {$X$};
\end{tikzpicture}%
\qquad\quad
\begin{tikzpicture}[baseline = (current bounding box.east)]
\draw[thick, arrows = {-Latex}] (0, 1.5) to (0, 1);
\draw[thick] (-0.5, 1) rectangle ++(1, -0.5) node[pos = 0.5] {$f$};
\draw[thick, arrows = {-Latex}] (0, 0.5) to (0, 0);
\node at (-0.5, 0.75) [left] {$f =\ $};
\node at (0, 1.5) [above] {$X$};
\node at (0, 0) [below] {$Y$};
\end{tikzpicture}%
\qquad\quad
\begin{tikzpicture}[baseline = (current bounding box.east)]
\draw[thick, arrows = {-Latex}] (0, 2.5) to (0, 2);
\draw[thick] (-0.5, 2) rectangle ++(1, -0.5) node[pos = 0.5] {$f$};
\draw[thick, arrows = {-Latex}] (0, 1.5) to (0, 1);
\draw[thick] (-0.5, 1) rectangle ++(1, -0.5) node[pos = 0.5] {$g$};
\draw[thick, arrows = {-Latex}] (0, 0.5) to (0, 0);
\node at (-0.5, 1.25) [left] {$g \circ f =\ $};
\end{tikzpicture}
\end{center}
Unfortunately, they tend to be rather boring. If we want to draw knots and links, we're going to need a bit more structure...
\begin{center}
\begin{tikzpicture}[baseline = (current bounding box.east), decoration = {
	markings,
	mark = at position 0.5 with {\arrow{Latex}}
}]
\draw[thick, postaction = {decorate}] (0, 0.5) to (0, 2);
\draw[thick, postaction = {decorate}] (1, 2) to (1, 0.5);
\draw[thick] (0, 0.5) arc(180:360:0.5);
\end{tikzpicture}%
\qquad\quad
\begin{tikzpicture}[baseline = (current bounding box.east), decoration = {
	markings,
	mark = at position 0.5 with {\arrow{Latex}}
}]
\draw[thick] (1, 1.5) arc(0:180:0.5);
\draw[thick, postaction = {decorate}] (0, 1.5) to (0, 0);
\draw[thick, postaction = {decorate}] (1, 0) to (1, 1.5);
\end{tikzpicture}%
\qquad\quad
\begin{tikzpicture}[baseline = (current bounding box.east)]
\begin{knot}[clip width = 5]
\strand[thick, arrows = {-Latex}, spath/arrow shortening = false] (0, 2) .. controls +(0.1, -1) and +(-0.1, 1) .. (1, 0.2);
\strand[thick, arrows = {-Latex}, spath/arrow shortening = false] (1, 2) .. controls +(-0.1, -1) and +(0.1, 1) .. (0, 0.2);
\end{knot}
\end{tikzpicture}%
\qquad\quad
\begin{tikzpicture}[baseline = (current bounding box.east)]
\begin{scope}[blend group = darken]
\begin{scope}[blend group = normal]
\fill[lightgray] (0.5, 1.5) .. controls +(-0.45, -0.25) .. (0, 1) .. controls +(0.05, -0.25) .. (0.5, 0.5) .. controls +(0.45, 0.25) .. (1, 1) .. controls +(-0.05, 0.25) .. (0.5, 1.5);
\end{scope}
\begin{scope}[blend group = normal]
\draw[thick] (0, 2) .. controls +(0, -0.5) and +(0, 0.5) .. (1, 1) .. controls +(0, -0.5) and +(0, 0.5) .. (0, 0);
\draw[thick] (1, 2) .. controls +(0, -0.5) and +(0, 0.5) .. (0, 1) .. controls +(0, -0.5) and +(0, 0.5) .. (1, 0);
\end{scope}
\end{scope}
\end{tikzpicture}
\end{center}
\end{frame}

\begin{frame}
\frametitle{Enter: The Second Dimension}
To place strings side by side, we define a \textcolor{structure}{monoidal structure}. That is, we have a product $\otimes$ with unit $\mathbbm{1}$, where $X \otimes Y$ denotes the object whose strands are the strands of $X$ and $Y$ placed parallel. Isotopy invariance implies that we must be able to ignore parentheses\textsuperscript{$\dagger$} and strands of $\mathbbm{1}$, and that morphisms should be able to freely move vertically along strands.
\begin{center}
\begin{tikzpicture}[baseline = (current bounding box.east)]
\draw[thick, arrows = {-Latex}] (0, 1.5) to (0, 0);
\node at (0, 1.5) [above] {$X \otimes Y$};
\node at (0, 0) [below] {$X \otimes Y$};

\node at (0.625, 0.75) {$=$};

\draw[thick, arrows = {-Latex}] (1.25, 1.5) to (1.25, 0);
\draw[thick, arrows = {-Latex}] (2, 1.5) to (2, 0);
\node at (1.25, 1.5) [above] {$X$};
\node at (1.25, 0) [below] {$X$};
\node at (2, 1.5) [above] {$Y$};
\node at (2, 0) [below] {$Y$};
\end{tikzpicture}%
\qquad\quad
\begin{tikzpicture}[baseline = (current bounding box.east)]
\draw[thick, arrows = {-Latex}] (0, 2.5) to (0, 2);
\draw[thick] (-0.5, 2) rectangle ++(1, -0.5) node[pos = 0.5] {$f$};
\draw[thick, arrows = {-Latex}] (0, 1.5) to (0, 0);
\draw[thick, arrows = {-Latex}] (1, 2.5) to (1, 1);
\draw[thick] (0.5, 1) rectangle ++(1, -0.5) node[pos = 0.5] {$g$};
\draw[thick, arrows = {-Latex}] (1, 0.5) to (1, 0);

\node at (1.75, 1.25) {$=$};

\draw[thick, arrows = {-Latex}] (2.5, 2.5) to (2.5, 1.5);
\draw[thick, arrows = {-Latex}] (3.5, 2.5) to (3.5, 1.5);
\draw[thick] (2, 1.5) rectangle ++(2, -0.5) node[pos = 0.5] {$f \otimes g$};
\draw[thick, arrows = {-Latex}] (2.5, 1) to (2.5, 0);
\draw[thick, arrows = {-Latex}] (3.5, 1) to (3.5, 0);

\node at (4.25, 1.25) {$=$};

\draw[thick, arrows = {-Latex}] (5, 2.5) to (5, 1);
\draw[thick] (4.5, 1) rectangle ++(1, -0.5) node[pos = 0.5] {$f$};
\draw[thick, arrows = {-Latex}] (5, 0.5) to (5, 0);
\draw[thick, arrows = {-Latex}] (6, 2.5) to (6, 2);
\draw[thick] (5.5, 2) rectangle ++(1, -0.5) node[pos = 0.5] {$g$};
\draw[thick, arrows = {-Latex}] (6, 1.5) to (6, 0);
\end{tikzpicture}
\end{center}
The usual tensor product gives $\textcat{Vec}$ a monoidal structure, where $\mathbbm{1} = \mathbb{C}$.
\end{frame}

\begin{frame}
\frametitle{Wiggling Strings}
To reproduce even the unknot, we're going to need cups and caps. Our strings are oriented though, so we'll be going up after passing through one!
\newline\newline
Let $X^{*}$ denote $X$ with its orientation reversed. Then our cups and caps may be represented by morphisms $\begin{tikzpicture}\draw[thick, arrows = {-Latex[length = 1mm, width = 1.5mm]}] (0, 0) arc(0:-190:0.25);\end{tikzpicture} : X^{*} \otimes X \to \mathbbm{1}$, $\begin{tikzpicture}\draw[thick, arrows = {-Latex[length = 1mm, width = 1.5mm]}] (0, 0) arc(0:190:0.25);\end{tikzpicture} : \mathbbm{1} \to X \otimes X^{*}$. To be invariant under isotopy, the following \textcolor{structure}{zig-zag identities} must hold:
\begin{equation*}
\begin{tikzpicture}[baseline = (current bounding box.east), decoration = {
	markings,
	mark = at position 0.5 with {\arrow{Latex}}
}]
\draw[thick, postaction = {decorate}] (2, 1.5) to (2, 0.5);
\draw[thick] (2, 0.5) arc(0:-180:0.5);
\draw[thick, postaction = {decorate}] (1, 0.5) to (1, 1);
\draw[thick] (1, 1) arc(0:180:0.5);
\draw[thick, arrows = {-Latex}] (0, 1) to (0, 0);
\node at (2, 1.5) [above] {$X$};
\node at (0, 0) [below] {$X$};
\end{tikzpicture} = \begin{tikzpicture}[baseline = (current bounding box.east), decoration = {
	markings,
	mark = at position 0.5 with {\arrow{Latex}}
}]
\draw[thick, arrows = {-Latex}] (3, 1.5) to (3, 0);
\node at (3, 1.5) [above] {$X$};
\node at (3, 0) [below] {$X$};
\end{tikzpicture},%
\qquad\quad
\begin{tikzpicture}[baseline = (current bounding box.east), decoration = {
	markings,
	mark = at position 0.5 with {\arrow{Latex}}
}]
\draw[thick, arrows = {-Latex}] (0, 0.5) to (0, 1.5);
\draw[thick] (1, 0.5) arc(0:-180:0.5);
\draw[thick, postaction = {decorate}] (1, 1) to (1, 0.5);
\draw[thick] (2, 1) arc(0:180:0.5);
\draw[thick, postaction = {decorate}] (2, 0) to (2, 1);
\node at (0, 1.5) [above] {$X^{*}$};
\node at (2, 0) [below] {$X^{*}$};
\end{tikzpicture} = \begin{tikzpicture}[baseline = (current bounding box.east), decoration = {
	markings,
	mark = at position 0.5 with {\arrow{Latex}}
}]
\draw[thick, arrows = {-Latex}] (3, 0) to (3, 1.5);
\node at (3, 1.5) [above] {$X^{*}$};
\node at (3, 0) [below] {$X^{*}$};
\end{tikzpicture}.
\end{equation*}
We call $X^{*}$ a \textcolor{structure}{left dual} for $X$. Swapping the orientations of cups and caps defines \textcolor{structure}{right duals}, ${^{*}X}$. A category is called \textcolor{structure}{rigid} if every object has both left and right duals, and \textcolor{structure}{pivotal} if these duals coincide in a ``natural'' way.
\end{frame}

\begin{frame}
\frametitle{Our First (Boring) Knot}
In a pivotal category, we can draw oriented circles:
\begin{equation*}
\begin{tikzpicture}[baseline = (current bounding box.east)]
\draw[thick, arrows = {-Latex}] (1, 2) arc(0:185:0.5);
\draw[thick, arrows = {-Latex}] (0, 1.5) to (0, 1);
\draw[thick, arrows = {-Latex}] (1, 1) to (1, 1.5);
\draw[thick, arrows = {-Latex}] (0, 0.5) arc(-180:5:0.5);
\node at (0, 1.5) [above] {$X$};
\node at (1, 1.5) [above] {$X^{*}$};
\node at (0, 0.5) [above] {$X$};
\node at (1, 0.5) [above] {${^{*}X}$};
\end{tikzpicture} =\ \begin{tikzpicture}[baseline = (current bounding box.east)]
\draw[thick, postaction = {decorate}, decoration = {markings, mark = at position 0 with {\arrow{Latex}}}] (4, 1.25) arc(0:360:1);
\end{tikzpicture}.
\end{equation*}
We will ask that $\End{\mathbbm{1}} \cong \mathbb{C}$ so we can identify links with numbers.
\newline\newline
In $\textcat{Vec}$, the duals of $V$ are given by the \textcolor{structure}{dual space}, $V^{*} \coloneqq \Hom{V, \mathbb{C}}$. Given a basis $\{v_i\}_{i=1}^{n}$ for $V$ with dual basis $\{v_i^{*}\}_{i=1}^{n}$ for $V^{*}$, we have $\begin{tikzpicture}\draw[thick, arrows = {-Latex[length = 1mm, width = 1.5mm]}] (0, 0) arc(0:-190:0.25);\end{tikzpicture} : v^{*} \otimes v \mapsto v^{*}(v)$ and $\begin{tikzpicture}\draw[thick, arrows = {-Latex[length = 1mm, width = 1.5mm]}] (0, 0) arc(0:190:0.25);\end{tikzpicture} : 1 \mapsto \sum_{i=1}^{n}{v_i \otimes v_i^{*}}$.
\newline\newline
The pivotal structure in $\textcat{Vec}$ is given by the canonical isomorphism between a vector space and its double dual, $v \mapsto (v^{*} \mapsto v^{*}(v))$.
\end{frame}

\begin{frame}
\frametitle{Enter: The Third Dimension}
Alexander tells us that every link is given by a braid capped off on both ends. To get the other knots, all we need to do now is define crossings.
\newline\newline
A \textcolor{structure}{braiding} is a collection of isomorphisms $X \otimes Y \to Y \otimes X$ that are compatible with the monoidal structure (object ``pairings'' don't matter) and which allow morphisms to be transported through them. For instance:
\begin{center}
\begin{tikzpicture}[baseline = (current bounding box.east)]
\begin{knot}[clip width = 5]
\strand[thick, arrows = {-Latex}, spath/arrow shortening = false] (0, 3) .. controls +(0.1, -0.5) and +(-0.1, 0.5) .. (1, 2.275);
\strand[thick, arrows = {-Latex}, spath/arrow shortening = false] (1, 3) .. controls +(-0.1, -0.5) and +(0.1, 0.5) .. (0, 2.275);
\strand[thick, arrows = {-Latex}, spath/arrow shortening = false] (1, 1.5) .. controls +(0.1, -0.5) and +(-0.1, 0.5) .. (2, 0.775);
\strand[thick, arrows = {-Latex}, spath/arrow shortening = false] (2, 1.5) .. controls +(-0.1, -0.5) and +(0.1, 0.5) .. (1, 0.775);
\end{knot}
\draw[thick, arrows = {-Latex}] (2, 3) to (2, 2);
\draw[thick, arrows = {-Latex}] (0, 1.5) to (0, 0.5);
\node at (0, 3) [above] {$X$};
\node at (1, 3) [above] {$Y$};
\node at (2, 3) [above] {$Z$};
\node at (0, 1.5) [above] {$Y$};
\node at (1, 1.5) [above] {$X$};
\node at (2, 1.5) [above] {$Z$};
\node at (0, 0) [above] {$Y$};
\node at (1, 0) [above] {$Z$};
\node at (2, 0) [above] {$X$};

\node at (2.5, 1.75) {$=$};

\begin{knot}[clip width = 5]
%\strand[thick, arrows = {-Latex}, spath/arrow shortening = false] (3.5, 3) .. controls +(0.1, -1) and +(-0.1, 1) .. (5, 0.775);
%\strand[thick, arrows = {-Latex}, spath/arrow shortening = false] (5, 3) .. controls +(-0.1, -1) and +(0.1, 1) .. (3.5, 0.775);
\strand[thick, arrows = {-Latex}, spath/arrow shortening = false] (3, 3) .. controls +(0.125, -1) and +(-0.125, 1) .. (5, 0.775);
\strand[thick, arrows = {-Latex}, spath/arrow shortening = false] (4, 3) .. controls +(-0.075, -1) and +(0.075, 1) .. (3, 0.775);
\strand[thick, arrows = {-Latex}, spath/arrow shortening = false] (5, 3) .. controls +(-0.075, -1) and +(0.075, 1) .. (4, 0.775);
\end{knot}
%\node at (3.5, 3) [above] {$X$};
%\node at (5, 3) [above] {$Y \otimes Z$};
%\node at (3.5, 0) [above] {$Y \otimes Z$};
%\node at (5, 0) [above] {$X$};
\node at (3, 3) [above] {$X$};
\node at (4, 3) [above] {$Y$};
\node at (5, 3) [above] {$Z$};
\node at (3, 0) [above] {$Y$};
\node at (4, 0) [above] {$Z$};
\node at (5, 0) [above] {$X$};
\end{tikzpicture},%
\qquad\quad
\begin{tikzpicture}[baseline = (current bounding box.east)]
\draw[thick] (-0.5, 3.5) rectangle ++(1, -0.5) node[pos = 0.5] {$f$};
\draw[thick] (1, 3.5) rectangle ++(1, -0.5) node[pos = 0.5] {$g$};
\begin{knot}[clip width = 5]
\strand[thick, arrows = {-Latex}, spath/arrow shortening = false] (0, 3) .. controls +(0.1, -1) and +(-0.1, 1) .. (1.5, 0.775);
\strand[thick, arrows = {-Latex}, spath/arrow shortening = false] (1.5, 3) .. controls +(-0.1, -1) and +(0.1, 1) .. (0, 0.775);
\end{knot}

\node at (2, 1.75) {$=$};

\begin{knot}[clip width = 5]
\strand[thick, arrows = {-Latex}, spath/arrow shortening = false] (2.5, 3) .. controls +(0.1, -1) and +(-0.1, 1) .. (4, 0.775);
\strand[thick, arrows = {-Latex}, spath/arrow shortening = false] (4, 3) .. controls +(-0.1, -1) and +(0.1, 1) .. (2.5, 0.775);
\end{knot}
\draw[thick] (2, 0.5) rectangle ++(1, -0.5) node[pos = 0.5] {$g$};
\draw[thick] (3.5, 0.5) rectangle ++(1, -0.5) node[pos = 0.5] {$f$};
\end{tikzpicture}.
\end{center}
The category $\textcat{Vec}$ is braided using the map $u \otimes v \mapsto v \otimes u$.
\end{frame}

\begin{frame}
\frametitle{The Reidemeister Moves}
Two knots are equivalent if and only if they are related by a sequence of {\em Reidemeister moves}. Does our graphical calculus give invariants of knots?
\begin{center}
\begin{tabular}{rl}
Type \textrm{II}: & \begin{tikzpicture}[baseline = (current bounding box.east)]
\begin{knot}[clip width = 5]
\strand[thick] (0, 2) .. controls +(0, -0.5) and +(0, 0.5) .. (1, 1);
\strand[thick] (1, 2) .. controls +(0, -0.5) and +(0, 0.5) .. (0, 1);
\strand[thick, arrows = {-Latex}, spath/arrow shortening = false] (1, 1) .. controls +(-0.1, -0.5) and +(0.1, 0.5) .. (0, 0.275);
\strand[thick, arrows = {-Latex}, spath/arrow shortening = false] (0, 1) .. controls +(0.1, -0.5) and +(-0.1, 0.5) .. (1, 0.275);
\end{knot}

\node at (1.5, 1) {$=$};

\draw[thick, arrows = {-Latex}] (2, 2) to (2, 0);
\draw[thick, arrows = {-Latex}] (3, 2) to (3, 0);
\end{tikzpicture}\\[4em]
Type \textrm{III}: & \begin{tikzpicture}[baseline = (current bounding box.east)]
\begin{knot}[clip width = 5]
\strand[thick] (0.5, 3) .. controls +(0, -0.5) and +(0, 0.5) .. (0, 2);
\strand[thick] (0, 3) .. controls +(0, -0.5) and +(0, 0.5) .. (0.5, 2);
\strand[thick] (0.5, 2) .. controls +(0, -0.5) and +(0, 0.5) .. (1, 1);
\strand[thick] (1, 2) .. controls +(0, -0.5) and +(0, 0.5) .. (0.5, 1);
\strand[thick, arrows = {-Latex}, spath/arrow shortening = false] (0, 1) .. controls +(0.05, -0.5) and +(-0.05, 0.5) .. (0.5, 0.275);
\strand[thick, arrows = {-Latex}, spath/arrow shortening = false] (0.5, 1) .. controls +(-0.05, -0.5) and +(0.05, 0.5) .. (0, 0.275);
\end{knot}
\draw[thick] (1, 3) to (1, 2);
\draw[thick] (0, 2) to (0, 1);
\draw[thick, arrows = {-Latex}] (1, 1) to (1, 0);

\node at (1.5, 1.5) {$=$};

\begin{knot}[clip width = 5]
\strand[thick] (2.5, 3) .. controls +(0, -0.5) and +(0, 0.5) .. (2, 2);
\strand[thick] (2, 3) .. controls +(0, -0.5) and +(0, 0.5) .. (2.5, 2);
\strand[thick, arrows = {-Latex}, spath/arrow shortening = false] (2, 2) .. controls +(0.05, -1) and +(-0.05, 1) .. (2.5, 0.225);
\strand[thick, arrows = {-Latex}, spath/arrow shortening = false] (2.5, 2) .. controls +(0.05, -1) and +(-0.05, 1) .. (3, 0.225);
\strand[thick, arrows = {-Latex}, spath/arrow shortening = false] (3, 2) .. controls +(-0.05, -1) and +(0.05, 1) .. (2, 0.225);
\end{knot}
\draw[thick] (3, 3) to (3, 2);

\node at (3.5, 1.5) {$=$};

\begin{knot}[clip width = 5]
\strand[thick] (4, 3) .. controls +(0, -1) and +(0, 1) .. (4.5, 1);
\strand[thick] (4.5, 3) .. controls +(0, -1) and +(0, 1) .. (5, 1);
\strand[thick] (5, 3) .. controls +(0, -1) and +(0, 1) .. (4, 1);
\strand[thick, arrows = {-Latex}, spath/arrow shortening = false] (5, 1) .. controls +(-0.05, -0.5) and +(0.05, 0.5) .. (4.5, 0.275);
\strand[thick, arrows = {-Latex}, spath/arrow shortening = false] (4.5, 1) .. controls +(0.05, -0.5) and +(-0.05, 0.5) .. (5, 0.275);
\end{knot}
\draw[thick, arrows = {-Latex}] (4, 1) to (4, 0);

\node at (5.5, 1.5) {$=$};

\begin{knot}[clip width = 5]
\strand[thick] (6.5, 3) .. controls +(0, -0.5) and +(0, 0.5) .. (7, 2);
\strand[thick] (7, 3) .. controls +(0, -0.5) and +(0, 0.5) .. (6.5, 2);
\strand[thick] (6, 2) .. controls +(0, -0.5) and +(0, 0.5) .. (6.5, 1);
\strand[thick] (6.5, 2) .. controls +(0, -0.5) and +(0, 0.5) .. (6, 1);
\strand[thick, arrows = {-Latex}, spath/arrow shortening = false] (7, 1) .. controls +(-0.05, -0.5) and +(0.05, 0.5) .. (6.5, 0.275);
\strand[thick, arrows = {-Latex}, spath/arrow shortening = false] (6.5, 1) .. controls +(0.05, -0.5) and +(-0.05, 0.5) .. (7, 0.275);
\end{knot}
\draw[thick] (6, 3) to (6, 2);
\draw[thick] (7, 2) to (7, 1);
\draw[thick, arrows = {-Latex}] (6, 1) to (6, 0);
\end{tikzpicture}
\end{tabular}
\end{center}
\end{frame}

\begin{frame}
\frametitle{Twists and Ribbons}
Unfortunately, the type \textrm{I} move does not hold in general. The reason is clear if we think of our strands as \textcolor{structure}{ribbons} rather than strings:
\begin{equation*}
\begin{tikzpicture}[baseline = (current bounding box.east)]
\begin{knot}[
	clip width = 5,
	only when rendering/.style = {
		draw = black,
		double = white,
		double distance = 6pt
	}
]
\strand[thick] (0, 2) -- (0, 1.5);
\strand[thick] (2, 1.5) arc (0:180:0.5);
\strand[thick] (0, 1.5) .. controls +(0, -0.5) and +(0, 0.5) .. (1, 0.5);
\strand[thick] (1, 1.5) .. controls +(0, -0.5) and +(0, 0.5) .. (0, 0.5);
\strand[thick] (2, 0.5) -- (2, 1.5);
\strand[thick] (0, 0.5) -- (0, 0);
\strand[thick] (1, 0.5) arc (180:360:0.5);
\end{knot}
\end{tikzpicture}\ =\ \begin{tikzpicture}[baseline = (current bounding box.east)]
\begin{scope}[blend group = darken]
\begin{scope}[blend group = normal]
\fill[lightgray] (3, 1.5) .. controls +(-2.7pt, -0.25) .. (3cm - 3pt, 1) .. controls +(0.3pt, -0.25) .. (3, 0.5) .. controls +(2.7pt, 0.25) .. (3cm + 3pt, 1) .. controls +(-0.3pt, 0.25) .. (3, 1.5);
\end{scope}
\begin{scope}[blend group = normal]
\draw[thick] (3cm - 3pt, 2) .. controls +(0, -0.5) and +(0, 0.5) .. (3cm + 3pt, 1) .. controls +(0, -0.5) and +(0, 0.5) .. (3cm - 3pt, 0);
\draw[thick] (3cm + 3pt, 2) .. controls +(0, -0.5) and +(0, 0.5) .. (3cm - 3pt, 1) .. controls +(0, -0.5) and +(0, 0.5) .. (3cm + 3pt, 0);
\end{scope}
\end{scope}
\end{tikzpicture}\ \neq\ \begin{tikzpicture}[baseline = (current bounding box.east)]
\draw[thick] (4cm - 3pt, 2) -- (4cm - 3pt, 0);
\draw[thick] (4cm + 3pt, 2) -- (4cm + 3pt, 0);
\end{tikzpicture}.
\end{equation*}
In a pivotal braided category, the diagram above defines a \textcolor{structure}{twist} $X \to X$. We call it a \textcolor{structure}{ribbon structure} if we can pull it out of zig-zags:
\begin{equation*}
\begin{tikzpicture}[baseline = (current bounding box.east), decoration = {
	markings,
	mark = at position 0.75 with {\arrow{Latex}}
}]
\draw[thick] (3, 0) -- (3, 1.125);
\draw[thick] (3, 1.125) arc (0:180:1.125);
\draw[thick] (2.25, 1.125) arc (0:180:0.375);
\begin{knot}[clip width = 5]
\strand[thick] (0.75, 1.125) .. controls +(0, -0.375) and +(0, 0.375) .. (1.5, 0.375);
\strand[thick] (1.5, 1.125) .. controls +(0, -0.375) and +(0, 0.375) .. (0.75, 0.375);
\end{knot}
\draw[thick, postaction = {decorate}] (2.25, 0.375) -- (2.25, 1.125);
\draw[thick] (1.5, 0.375) arc (180:360:0.375);
\draw[thick] (0, 0.375) arc (180:360:0.375);
\draw[thick, arrows = {-Latex}] (0, 0.375) -- (0, 2.25);
\end{tikzpicture}\ =\ \begin{tikzpicture}[baseline = (current bounding box.east), decoration = {
	markings,
	mark = at position 0.75 with {\arrow{Latex}}
}]
\draw[thick] (4, 0) -- (4, 0.75);
\draw[thick] (5.5, 0.75) arc (360:180:0.375);
\draw[thick, postaction = {decorate}] (5.5, 1.5) -- (5.5, 0.75);
\begin{knot}[clip width = 5]
\strand[thick] (4, 1.5) .. controls +(0, -0.375) and +(0, 0.375) .. (4.75, 0.75);
\strand[thick] (4.75, 1.5) .. controls +(0, -0.375) and +(0, 0.375) .. (4, 0.75);
\end{knot}
\draw[thick] (4.75, 1.5) arc (180:0:0.375);
\draw[thick, arrows = {-Latex}] (4, 1.5) -- (4, 2.25);
\end{tikzpicture}.
\end{equation*}
\end{frame}

\begin{frame}
\frametitle{Invariants of Knots and 3-Manifolds}
Note that twists satisfy the following \textcolor{structure}{modified} type \textrm{I} Reidemeister move:
\begin{equation*}
\begin{tikzpicture}[baseline = (current bounding box.east), decoration = {
	markings,
	mark = at position 0.75 with {\arrow{Latex}}
}]
\draw[thick] (0, 3.375) -- (0, 3);
\draw[thick] (1.5, 3) arc (0:180:0.375);
\begin{knot}[clip width = 5]
\strand[thick] (0, 3) .. controls +(0, -0.375) and +(0, 0.375) .. (0.75, 2.25);
\strand[thick] (0.75, 3) .. controls +(0, -0.375) and +(0, 0.375) .. (0, 2.25);
\end{knot}
\draw[thick, postaction = {decorate}] (1.5, 2.25) -- (1.5, 3);
\draw[thick] (0.75, 2.25) arc (180:360:0.375);
\draw[thick, postaction = {decorate}] (0, 2.25) -- (0, 1.125);
\draw[thick] (1.5, 1.125) arc (0:180:0.375);
\begin{knot}[clip width = 5]
\strand[thick] (0.75, 1.125) .. controls +(0, -0.375) and +(0, 0.375) .. (0, 0.375);
\strand[thick] (0, 1.125) .. controls +(0, -0.375) and +(0, 0.375) .. (0.75, 0.375);
\end{knot}
\draw[thick, postaction = {decorate}] (1.5, 0.375) -- (1.5, 1.125);
\draw[thick] (0.75, 0.375) arc (180:360:0.375);
\draw[thick, arrows = {-Latex}] (0, 0.375) -- (0, 0);
\end{tikzpicture}\ =\ \begin{tikzpicture}[baseline = (current bounding box.east)]
\draw[thick, arrows = {-Latex}] (4, 3.375) -- (4, 0);
\end{tikzpicture}.
\end{equation*}
The numbers associated to link diagrams coloured by any object give invariants of \textcolor{structure}{framed, oriented links}. This also gives invariants of closed, orientable, connected 3-manifolds by a theorem of Lickorish and Wallace: all such manifolds can be obtained by Dehn surgery on a framed link in $S^3$.
\end{frame}

\begin{frame}
\frametitle{Forgetting the Framing}
It turns out that ribbon twists of simple objects are scalings of the identity:
\begin{equation*}
\begin{tikzpicture}[baseline = (current bounding box.east), decoration = {
	markings,
	mark = at position 0.75 with {\arrow{Latex}}
}]
\draw[thick] (0, 2.25) -- (0, 1.5);
\draw[thick] (1.5, 1.5) arc (0:180:0.375);
\begin{knot}[clip width = 5]
\strand[thick] (0, 1.5) .. controls +(0, -0.375) and +(0, 0.375) .. (0.75, 0.75);
\strand[thick] (0.75, 1.5) .. controls +(0, -0.375) and +(0, 0.375) .. (0, 0.75);
\end{knot}
\draw[thick, postaction = {decorate}] (1.5, 0.75) -- (1.5, 1.5);
\draw[thick] (0.75, 0.75) arc (180:360:0.375);
\draw[thick, arrows = {-Latex}] (0, 0.75) -- (0, 0);
\node at (0, 2.25) [above] {$X$};
\node at (0, 0) [below] {$X$};
\end{tikzpicture}\ =\ \theta_X\begin{tikzpicture}[baseline = (current bounding box.east)]
\draw[thick, arrows = {-Latex}] (0, 2.25) -- (0, 0);
\node at (0, 2.25) [above] {$X$};
\node at (0, 0) [below] {$X$};
\end{tikzpicture}.
\end{equation*}
For any link diagram $L$ coloured by $X$, $\theta_X^{-\Wr(L)}L$ gives an invariant of $L$ as an \textcolor{structure}{unframed link}. Here, $\Wr(L)$ is the \textcolor{structure}{writhe} of $L$ (the number of positive crossings minus the number of negative crossings).
\newline\newline
While one can always forget the framing, forgetting the orientation is trickier. It is necessary but not sufficient for the object to be self-dual!
\end{frame}

%%%%%%%%%%%%%%%%%%%%%%%%%%%%%
% Modular Tensor Categories %
%%%%%%%%%%%%%%%%%%%%%%%%%%%%%

%\section{Modular Tensor Categories}
%
%\begin{frame}
%\centerline{\Huge\textcolor{structure}{\underline{Modular Tensor Categories}}}
%\end{frame}
%
%\begin{frame}
%\frametitle{Modular Tensor Categories}
%In a ribbon fusion category, we can define two matrices that together have quite remarkable properties. For any simple objects $X$ and $Y$, define
%\begin{equation*}
%s_{X,Y} = \begin{tikzpicture}[baseline = (current bounding box.east), decoration = {
%	markings,
%	mark = at position 0.5 with {\arrow{Latex}}
%}]
%%\draw[thick, arrows = {-Latex}] ([shift = (312.5:0.75)] -0.5625, 0.75) arc (312.5:-35.5:0.75);
%%\draw[thick, arrows = {-Latex}] ([shift = (-215.5:0.75)] 0.5625, 0.75) arc (-215.5:132.5:0.75);
%%\node at (-0.5625, 1.75) {$X$};
%%\node at (0.5625, 1.75) {$Y$};
%
%\draw[thick] (0, 3.5) arc (180:0:0.5);
%\draw[thick] (3, 3.5) arc (0:180:0.5);
%\draw[thick, postaction = {decorate}] (0, 0.5) -- (0, 3.5);
%\draw[thick, postaction = {decorate}] (3, 0.5) -- (3, 3.5);
%\begin{knot}[clip width = 4]
%\strand[thick] (1, 3) .. controls +(0, -0.5) and +(0, 0.5) .. (2, 2);
%\strand[thick] (2, 3) .. controls +(0, -0.5) and +(0, 0.5) .. (1, 2);
%\strand[thick] (1, 2) .. controls +(0, -0.5) and +(0, 0.5) .. (2, 1);
%\strand[thick] (2, 2) .. controls +(0, -0.5) and +(0, 0.5) .. (1, 1);
%\end{knot}
%\draw[thick] (1, 0.5) arc (0:-180:0.5);
%\draw[thick] (2, 0.5) arc (180:360:0.5);
%\node at (1, 3) [above] {$X$};
%\node at (2, 3) [above] {$Y$};
%\node at (1, 1) [below] {$X$};
%\node at (2, 1) [below] {$Y$};
%\end{tikzpicture},%
%\qquad\quad
%\begin{tikzpicture}[baseline = (current bounding box.east), decoration = {
%	markings,
%	mark = at position 0.6 with {\arrow{Latex}}
%}]
%\draw[thick] (0, 2) -- (0, 1.5);
%\draw[thick] (2, 1.5) arc (0:180:0.5);
%\begin{knot}[clip width = 5]
%\strand[thick] (0, 1.5) .. controls +(0, -0.5) and +(0, 0.5) .. (1, 0.5);
%\strand[thick] (1, 1.5) .. controls +(0, -0.5) and +(0, 0.5) .. (0, 0.5);
%\end{knot}
%\draw[thick, postaction = {decorate}] (2, 0.5) -- (2, 1.5);
%\draw[thick, arrows = {-Latex}] (0, 0.5) -- (0, 0);
%\draw[thick] (1, 0.5) arc (180:360:0.5);
%\node at (0, 2) [above] {$X$};
%\node at (0, 0) [below] {$X$};
%\end{tikzpicture} = t_X\begin{tikzpicture}[baseline = (current bounding box.east), decoration = {
%	markings,
%	mark = at position 0.6 with {\arrow{Latex}}
%}]
%\draw[thick, arrows = {-Latex}] (3.5, 2) -- (3.5, 0);
%\node at (3.5, 2) [above] {$X$};
%\node at (3.5, 0) [below] {$X$};
%\end{tikzpicture}.
%\end{equation*}
%The $S$- and $T$-matrices have rows and columns indexed by simple objects and entries $s_{X,Y}, t_X$
%\end{frame}

\begin{frame}
\frametitle{Examples at Long Last}
\begin{example}[Vector Spaces]
For $\mathbb{C}^n$ in $\textcat{Vec}$, we get $L \mapsto n^c$ for a link $L$ with $c$ components. Wow!!
\end{example}
\begin{example}[Graded Vector Spaces]
Consider the category $\textcat{Vec}_G$ of vector spaces \textcolor{structure}{graded} by $G$. This category has 1D simple objects $\mathbb{C}_g$ for each $g \in G$, which satisfy $\mathbb{C}_g \otimes \mathbb{C}_h = \mathbb{C}_{gh}$ and $\mathbb{C}_g^{*} = \mathbb{C}_{g^{-1}}$. The braidings are given by \textcolor{structure}{quadratic forms} on $G$. For instance, if $G = \mathbb{Z}/2\mathbb{Z} = \{\mathbbm{1}, g\}$, we can define $\begin{tikzpicture}\draw[thick, arrows = {-Latex[length = 1mm, width = 1.5mm]}] (0, 0) arc(0:-190:0.25);\end{tikzpicture} = i \cdot \id_{\mathbb{C}_{\mathbbm{1}}} = \begin{tikzpicture}\draw[thick, arrows = {-Latex[length = 1mm, width = 1.5mm]}] (0, 0) arc(0:190:0.25);\end{tikzpicture}$ and
\begin{align*}
\begin{split}
\begin{tikzpicture}[baseline = (current bounding box.east)]
\begin{knot}[clip width = 5, background color = bg]
\strand[thick, arrows = {-Latex}, spath/arrow shortening = false] (0, 1) .. controls +(0.05, -0.5) and +(-0.05, 0.5) .. (0.5, 0);
\strand[thick, arrows = {-Latex}, spath/arrow shortening = false] (0.5, 1) .. controls +(-0.05, -0.5) and +(0.05, 0.5) .. (0, 0);
\end{knot}
\end{tikzpicture} = i \cdot \id_{\mathbb{C}_{\mathbbm{1}}}, \quad \begin{tikzpicture}[baseline = (current bounding box.east)]
\draw[thick] (0, 1.5) -- (0, 1.25);
\draw[thick] (1, 1.25) arc (0:180:0.25);
\begin{knot}[clip width = 5, background color = bg]
\strand[thick] (0, 1.25) .. controls +(0, -0.5) and +(0, 0.5) .. (0.5, 0.25);
\strand[thick] (0.5, 1.25) .. controls +(0, -0.5) and +(0, 0.5) .. (0, 0.25);
\end{knot}
\draw[thick] (1, 0.25) -- (1, 1.25);
\draw[thick, arrows = {-Latex}] (0, 0.25) -- (0, 0);
\draw[thick] (0.5, 0.25) arc (180:360:0.25);
\end{tikzpicture} = i \cdot \id_{\mathbb{C}_g}.
\end{split}
\end{align*}
The framed link invariant is $L \mapsto i^{\Wr(L)}$, so the unframed invariant is trivial. In this example, $\mathbb{C}_g$ is self-dual, but the orientation cannot be removed!
\end{example}
\end{frame}

\begin{frame}
\frametitle{The Important Example}
\begin{example}[Jones Polynomial]
The category $\textcat{Rep}(U_q(\mathfrak{sl}_2))$ for $q \neq \pm 1$ a root of unity is ribbon. It has a 2D simple object $V$ that is self-dual with $\begin{tikzpicture}\draw[thick] (0, 0) arc(0:-180:0.25);\end{tikzpicture} = \begin{bmatrix}
0 & 1 & -q^{-1} & 0
\end{bmatrix}$ and $\begin{tikzpicture}\draw[thick] (0, 0) arc(0:180:0.25);\end{tikzpicture} = \begin{bmatrix}
0 & -q & 1 & 0
\end{bmatrix}^T$. The braiding and ribbon twist for $V$ are given by
\begin{align*}
\begin{split}
\begin{tikzpicture}[baseline = (current bounding box.east)]
\begin{knot}[clip width = 5, background color = bg]
\strand[thick] (0, 1) .. controls +(0, -0.5) and +(0, 0.5) .. (0.5, 0);
\strand[thick] (0.5, 1) .. controls +(0, -0.5) and +(0, 0.5) .. (0, 0);
\end{knot}
\end{tikzpicture} = q^{-1/2}\begin{bmatrix}
q & 0 & 0 & 0 \\
0 & 0 & 1 & 0 \\
0 & 1 & q - q^{-1} & 0 \\
0 & 0 & 0 & q
\end{bmatrix}, \quad \begin{tikzpicture}[baseline = (current bounding box.east)]
\draw[thick] (0, 1.5) -- (0, 1.25);
\draw[thick] (1, 1.25) arc (0:180:0.25);
\begin{knot}[clip width = 5, background color = bg]
\strand[thick] (0, 1.25) .. controls +(0, -0.5) and +(0, 0.5) .. (0.5, 0.25);
\strand[thick] (0.5, 1.25) .. controls +(0, -0.5) and +(0, 0.5) .. (0, 0.25);
\end{knot}
\draw[thick] (1, 0.25) -- (1, 1.25);
\draw[thick] (0, 0.25) -- (0, 0);
\draw[thick] (0.5, 0.25) arc (180:360:0.25);
\end{tikzpicture} = -q^{3/2} \cdot \id_V.
\end{split}
\end{align*}
It's an easy exercise to show that this satisfies the following \textcolor{structure}{skein relation}:
\begin{align*}
\begin{split}
\begin{tikzpicture}[baseline = (current bounding box.east)]
\begin{knot}[clip width = 5, background color = bg]
\strand[thick] (0, 1) .. controls +(0, -0.5) and +(0, 0.5) .. (0.5, 0);
\strand[thick] (0.5, 1) .. controls +(0, -0.5) and +(0, 0.5) .. (0, 0);
\end{knot}
\end{tikzpicture} = q^{1/2}\begin{tikzpicture}[baseline = (current bounding box.east)]
\draw[thick] (0, 1) -- (0, 0);
\draw[thick] (0.5, 1) -- (0.5, 0);
\end{tikzpicture} + q^{-1/2}\begin{tikzpicture}[baseline = (current bounding box.east)]
\draw[thick] (0.5, 1) arc(0:-180:0.25);
\draw[thick] (0.5, 0) arc(0:180:0.25);
\end{tikzpicture}.
\end{split}
\end{align*}
I've implicitly chosen here the structure that removes the orientation.
\end{example}
\end{frame}

\begin{frame}
\frametitle{The Jones Polynomial Generalized}
\begin{example}[HOMFLYPT Polynomial]
More generally, $\textcat{Rep}(U_q(\mathfrak{sl}_n))$ is ribbon. The standard ($n$-dimensional) representation $V_n$ has the same caps and cups as $\textcat{Vec}$, but letting $v_{n-1}, v_{n-3}, \dots, v_{3-n}, v_{1-n}$ be a basis for $V_n$, its braiding is given by
\begin{align*}
\begin{split}
\begin{tikzpicture}[baseline = (current bounding box.east)]
\begin{knot}[clip width = 5, background color = bg]
\strand[thick, arrows = {-Latex}, spath/arrow shortening = false] (0, 1) .. controls +(0.05, -0.5) and +(-0.05, 0.5) .. (0.5, 0);
\strand[thick, arrows = {-Latex}, spath/arrow shortening = false] (0.5, 1) .. controls +(-0.05, -0.5) and +(0.05, 0.5) .. (0, 0);
\end{knot}
\end{tikzpicture} : v_i \otimes v_j \mapsto q^{-1/n}\begin{cases}
qv_i \otimes v_j, & \textnormal{if $i = j$;} \\
v_j \otimes v_i, & \textnormal{if $i > j$;} \\
v_j \otimes v_i + (q - q^{-1})v_i \otimes v_j, & \textnormal{if $i < j$.}
\end{cases}
\end{split}
\end{align*}
The ribbon structure is $q^{n - 1/n} \cdot \id_{V_n}$. This time, we have skein relation
\begin{align*}
\begin{split}
q^{1/n}\begin{tikzpicture}[baseline = (current bounding box.east)]
\begin{knot}[clip width = 5, background color = bg]
\strand[thick, arrows = {-Latex}, spath/arrow shortening = false] (0, 1) .. controls +(0.05, -0.5) and +(-0.05, 0.5) .. (0.5, 0);
\strand[thick, arrows = {-Latex}, spath/arrow shortening = false] (0.5, 1) .. controls +(-0.05, -0.5) and +(0.05, 0.5) .. (0, 0);
\end{knot}
\end{tikzpicture} - q^{-1/n}\begin{tikzpicture}[baseline = (current bounding box.east)]
\begin{knot}[clip width = 5, background color = bg]
\strand[thick, arrows = {-Latex}, spath/arrow shortening = false] (0.5, 1) .. controls +(-0.05, -0.5) and +(0.05, 0.5) .. (0, 0);
\strand[thick, arrows = {-Latex}, spath/arrow shortening = false] (0, 1) .. controls +(0.05, -0.5) and +(-0.05, 0.5) .. (0.5, 0);
\end{knot}
\end{tikzpicture} = (q - q^{-1})\begin{tikzpicture}[baseline = (current bounding box.east)]
\draw[thick, arrows = {-Latex}] (0, 1) -- (0, 0);
\draw[thick, arrows = {-Latex}] (0.5, 1) -- (0.5, 0);
\end{tikzpicture}.
\end{split}
\end{align*}
This defines the \textcolor{structure}{$n$-specialization} of the HOMFLYPT polynomial.
\end{example}
%We can play this game for other simple representations and Lie algebras $\mathfrak{g}$.
%\begin{example}[Coloured Jones Polynomial]
%If $\mathfrak{g} = \mathfrak{sl}_2$, the simple $n$-dimensional representations for $n > 2$ give coloured Jones polynomials.
%\end{example}
%\begin{example}[HOMFLYPT Polynomial]
%If $\mathfrak{g} = \mathfrak{sl}_n$, the fundamental representations give the HOMFLYPT polynomial.
%\end{example}
%\begin{example}[Kauffman Polynomial]
%If $\mathfrak{g} = \mathfrak{so}_n$, the fundamental representations give the Kauffman polynomial.
%\end{example}
\end{frame}

%%%%%%%%%%%%%%
% References %
%%%%%%%%%%%%%%

\section*{References}

\begin{frame}[allowframebreaks]
\frametitle{References}
\footnotesize{
\printbibliography[heading = none]
}
\end{frame}

%%%%%%%%%%%%%
% Thank you %
%%%%%%%%%%%%%

\begin{frame}
\centerline{\Huge\textcolor{structure}{\underline{Thank you for listening!}}}
\end{frame}

\end{document} 