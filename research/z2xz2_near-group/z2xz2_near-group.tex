%%%%%%%%%%%%%%%%%%%%%%%%%%%%%%%%%%%%%%%%
%                                      %
% Random LaTeX Template                %
%                                      %
% Created by Thomas Dunmore.           %
% Last updated 2025/10/02.             %
%                                      %
%%%%%%%%%%%%%%%%%%%%%%%%%%%%%%%%%%%%%%%%

%%%%%%%%%
% SETUP %
%%%%%%%%%

\documentclass[12pt, reqno]{amsart}
\usepackage[a4paper, total={18cm, 22cm}, centering]{geometry}

%\usepackage[utf8]{inputenc}
%\usepackage[T1]{fontenc}
%\usepackage{lmodern}
\usepackage[english]{babel}
\usepackage{csquotes}
\usepackage{amsmath, amsfonts, amsthm, amssymb, mathrsfs}
\usepackage{graphicx}
\usepackage{hyperref}
\usepackage[usenames, dvipsnames]{color}

% Center title.
\usepackage{titling}
\renewcommand\maketitlehooka{\null\mbox{}\vfill}
\renewcommand\maketitlehookd{\vfill\null}
% Title block on first page.
\title{
	\normalfont \fontsize{10.95}{13.14}
	\LARGE Formal Codegrees in Multifusion Categories \\[-\linespacing]
	\rule{0.9\linewidth}{1pt} \\[0.5cm]
	\large Thomas Dunmore z5256305 \\[12pt]
	\small\today
}

% Custom headers and footers.
\usepackage{fancyhdr}
\pagestyle{fancy}
\fancyhead[L]{}
\fancyhead[C]{}
\fancyhead[R]{}
\fancyfoot[L]{}
\fancyfoot[C]{\thepage}
\fancyfoot[R]{}
\renewcommand{\headrulewidth}{0pt}
\renewcommand{\footrulewidth}{0pt}
\setlength{\headheight}{14pt}
\setlength{\headsep}{36pt}
\setlength{\footskip}{14pt}

% Bibliography.
\usepackage[%
	backend = biber,
	% BibLaTeX-Math package: https://github.com/konn/biblatex-math
	style = math-alphabetic,
	giveninits = true,
	dashed = false,
	url = false,
	doi = false,
	sorting = anyvt,
	minalphanames = 3,
	maxalphanames = 4
]{biblatex}
% Use the default font size for the bibliography.
\renewcommand*{\bibfont}{\normalsize}
% Use title case rather than sentence case for references.
\DeclareFieldFormat{titlecase}{#1}
% Put last names before first names.
\DeclareNameAlias{default}{family-given}
% Used for articles with appendices written by other authors.
\NewBibliographyString{bywithappendix}
\DefineBibliographyStrings{english}{
	bywithappendix = {with an appendix by}
}
% Specify the bibliography data file to use.
%\addbibresource{references.bib}


%%%%%%%%%%%%%%
% FORMATTING %
%%%%%%%%%%%%%%

% Add dotted lines after table of contents entries.
\makeatletter
\def\@tocline#1#2#3#4#5#6#7{\relax
	\ifnum #1>\c@tocdepth
	\else
		\par \addpenalty\@secpenalty\addvspace{#2}
		\begingroup \hyphenpenalty\@M
		\@ifempty{#4}{
			\@tempdima\csname r@tocindent\number#1\endcsname\relax
		}{
			\@tempdima#4\relax
		}
		\parindent\z@ \leftskip#3\relax \advance\leftskip\@tempdima\relax
		\rightskip\@pnumwidth plus4em \parfillskip-\@pnumwidth
		#5\leavevmode\hskip-\@tempdima #6\nobreak\relax
		\ifnum#1<0\hfill\else\dotfill\fi\hbox to\@pnumwidth{\@tocpagenum{#7}}\par
		\nobreak
		\endgroup
\fi}
\makeatother
% Align table of contents items horizontally.
\makeatletter
\renewcommand{\l@section}{\@tocline{1}{0pt}{0pt}{}{}}
\renewcommand{\l@subsection}{\@tocline{2}{0pt}{10pt}{}{}}
\let\tocsubsection\tocsection
\renewcommand{\l@subsubsection}{\@tocline{3}{0pt}{20pt}{}{}}
\let\tocsubsubsection\tocsection
\makeatother

% Section formatting.
\makeatletter
\def\section{\@startsection{section}{1}\z@{0pt}{0.5\linespacing}{\LARGE\scshape}}
\def\subsection{\@startsection{subsection}{2}\z@{0pt}{0.5\linespacing}{\large}}
\def\subsubsection{\@startsection{subsubsection}{3}\z@{0pt}{0.5\linespacing}{\itshape}}
\makeatother

% Apply section formatting to numbers.
\makeatletter
\def\@seccntformat#1{
	\protect\@secnumfont
	\expandafter\protect\csname format#1\endcsname
	\csname the#1\endcsname
	\protect\@secnumpunct
}
% The hack above doesn't work for boldface, so we need to specify it manually.
%\newcommand{\formatsection}{\bfseries\boldmath}
%\newcommand{\formatsubsection}{\bfseries\boldmath}
%\newcommand{\formatsubsubsection}{\bfseries\boldmath}
\makeatother

% Add fake sections and subsections to table of contents.
\newcommand{\fakesection}[1]{%
	\par\refstepcounter{section} % Increase section counter
	\sectionmark{#1} % Add section mark (header)
	\addcontentsline{toc}{section}{\protect\numberline{\thesection}#1} % Add section to ToC
}
\newcommand{\fakesubsection}[1]{%
	\par\refstepcounter{subsection} % Increase subsection counter
	\subsectionmark{#1} % Add subsection mark (header)
	\addcontentsline{toc}{subsection}{\protect\numberline{\thesubsection}#1} % Add subsection to ToC
}

% Remove unspecified periods in section titles.
\makeatletter
\let\@addpunct\@gobble
\makeatother

% Automatically add a blank line between paragraphs.
%\setlength{\parskip}{\baselineskip}

% Number equations according to the sections they appear in.
\numberwithin{equation}{section}
%\numberwithin{equation}{subsection}
%\numberwithin{equation}{subsubsection}

% More space between equations.
%\AtBeginDocument{\addtolength{\jot}{10pt}}
% Remove the ugly random spacing that LaTeX likes to add.
\raggedbottom

% More table spacing.
% Don't use this if you want piecewise functions.
%\setlength{\extrarowheight}{5pt}

% Better double horizontal ruling for tables.
\usepackage{hhline}

% Allows for table cells that contain line breaks.
\newcommand{\specialcell}[2][c]{
    \begin{tabular}[#1]{@{}l@{}}#2\end{tabular}
}

% Table colours.
\usepackage[table, xcdraw]{xcolor}
\usepackage{colortbl}
\definecolor{ColourBlack}{HTML}{000000}
\definecolor{ColourWhite}{HTML}{FFFFFF}
\definecolor{ColourTableGrey}{HTML}{EDEDED}
\definecolor{Grey}{gray}{0.92}


%%%%%%%%%%%%%%%%
% ENVIRONMENTS %
%%%%%%%%%%%%%%%%

% Define commands for sections that support equation numbering.
\newcommand*{\problem}[2][]{
	\ifthenelse{\equal{#1}{}}
		{\section*{#2}\refstepcounter{section}}
		{\section*{\texorpdfstring{#2}{#1}}\refstepcounter{section}}
}
\newcommand*{\subproblem}[2][]{
	\ifthenelse{\equal{#1}{}}
		{\subsection*{#2}\refstepcounter{subsection}}
		{\subsection*{\texorpdfstring{#2}{#1}}\refstepcounter{subsection}}
}
\newcommand*{\subsubproblem}[2][]{
	\ifthenelse{\equal{#1}{}}
		{\subsubsection*{#2}\refstepcounter{subsubsection}}
		{\subsubsection*{\texorpdfstring{#2}{#1}}\refstepcounter{subsubsection}}
}

% Customizable enumerate list labels.
\usepackage{enumerate}
\usepackage[shortlabels]{enumitem}

% Theorem environments.
\newtheoremstyle{plainspace}{-\topsep}{-\topsep}{\itshape}{}{\bfseries}{.}{0.5em}{}
\theoremstyle{plainspace}
\newtheorem{theorem}{Theorem}[section]
\newtheorem{lemma}[theorem]{Lemma}
\newtheorem{corollary}[theorem]{Corollary}
\newtheorem{proposition}[theorem]{Proposition}
\newtheorem{conjecture}[theorem]{Conjecture}

\newtheoremstyle{definitionspace}{-\topsep}{-\topsep}{}{}{\bfseries}{.}{0.5em}{}
\theoremstyle{definitionspace}
\newtheorem{definition}[theorem]{Definition}
\newtheorem{example}[theorem]{Example}
\newtheorem{question}[theorem]{Question}

\newtheoremstyle{remarkspace}{-\topsep}{-\topsep}{}{}{\itshape}{.}{0.5em}{}
\theoremstyle{remarkspace}
\newtheorem{remark}[theorem]{Remark}
\newtheorem{notation}[theorem]{Notation}

% Proof environment formatting.
\renewenvironment{proof}{{\noindent\textbf{Proof.}}}{\null\hfill\qedsymbol}

% Modified matrix environments for augmented matrices.
\makeatletter
\renewcommand*\env@matrix[1][*\c@MaxMatrixCols c]{%
	\hskip -\arraycolsep
	\let\@ifnextchar\new@ifnextchar
	\array{#1}
}
\makeatother

% Modified cases environment for column alignment.
\makeatletter
\renewenvironment{cases}[1][l]{\matrix@check\cases\env@cases{#1}}{\endarray\right.}
\def\env@cases#1{%
	\let\@ifnextchar\new@ifnextchar
	\left\lbrace\def\arraystretch{1.2}%
	\array{@{}#1@{\quad}l@{}}}
\makeatother


%%%%%%%%%%%%
% COMMANDS %
%%%%%%%%%%%%

% Section title bar.
\newcommand{\sectionbar}[4]{
	\noindent\\[#3\linespacing] \rule{#1\linewidth}{#2} \\[#4\linespacing]
}
\newcommand{\sectiondiv}[4]{
	\noindent\\[#3\linespacing] \centerline{\rule{#1\linewidth}{#2}} \\[#4\linespacing]
}

% Negative horizontal phantom.
\newcommand{\nhphantom}[1]{\sbox0{#1}\hspace{-\the\wd0}}


%%%%%%%%%%%%
% NOTATION %
%%%%%%%%%%%%

% Adds integral notation like \oiint.
\usepackage{esint}
% Blackboard bold symbols.
\usepackage{bbm}

\usepackage{accents}
% Tilde notation for vectors.
\newcommand{\ut}[1]{\underaccent{\tilde}{#1}}
% Arrow notation for vectors.
\usepackage{harpoon}
% Dirac bra-ket notation for quantum states.
\usepackage{braket}

% Differential formatting.
\usepackage{ifthen}
\usepackage{etoolbox}
\newcommand*{\ndiff}[1]{\mathrm{d}#1}
\newcommand*{\sdiff}[1]{\mathop{}\!\ndiff{#1}}
\newcommand{\rdiff}[3][]{
	\ifthenelse{\equal{#1}{}}
		{\frac{\mathrm{d}#2}{\mathrm{d}#3}}
		{\frac{\mathrm{d}^{#1}#2}{\forcsvlist\ndiff{#3}}}
}
\newcommand*{\npiff}[1]{\mathrm{\partial}#1}
\newcommand*{\spiff}[1]{\mathop{}\!\npiff{#1}}
\newcommand{\rpiff}[3][]{
	\ifthenelse{\equal{#1}{}}
		{\frac{\mathrm{\partial}#2}{\mathrm{\partial}#3}}
		{\frac{\mathrm{\partial}^{#1}#2}{\forcsvlist\npiff{#3}}}
}
% Inexact differential for physics.
\newcommand*{\dbar}[1]{\mathop{}\!\mathrm{\dj}#1}

% Metrics, inner products and norms.
\usepackage{mathtools}
\DeclarePairedDelimiter{\abs}{\lvert}{\rvert}
\DeclarePairedDelimiter{\inprod}{\langle}{\rangle}
\DeclarePairedDelimiter{\norm}{\lVert}{\rVert}
% This is used if we want an empty norm. 
\newcommand{\blank}{{}\cdot{}}

% Function notation.
\newcommand{\id}{\textup{id}}
\newcommand{\coker}{\textup{coker}}
\newcommand{\im}{\textup{im}}
\newcommand{\ev}{\textup{ev}}
\newcommand{\coev}{\textup{coev}}

% Category font.
\newcommand{\mathcat}[1]{\mathcal{#1}}

% Category theory notation.
\newcommand{\Ob}{\textup{Ob}}
\newcommand{\Hom}[2][]{
	\ifthenelse{\equal{#2}{}}
		{\textup{Hom}_{#1}}
		{\textup{Hom}_{#1}\!\left(#2\right)}
}
\newcommand{\IntHom}[2][]{
	\ifthenelse{\equal{#2}{}}
		{\underline{\textup{Hom}}_{#1}}
		{\underline{\textup{Hom}}_{#1}\!\left(#2\right)}
}
\newcommand{\End}[2][]{
	\ifthenelse{\equal{#2}{}}
		{\textup{End}_{#1}}
		{\textup{End}_{#1}\!\left(#2\right)}
}
\newcommand{\IntEnd}[2][]{
	\ifthenelse{\equal{#2}{}}
		{\underline{\textup{End}}_{#1}}
		{\underline{\textup{End}}_{#1}\!\left(#2\right)}
}
\newcommand{\Fun}[2][]{
	\ifthenelse{\equal{#2}{}}
		{\textcat{Fun}_{#1}}
		{\textcat{Fun}_{#1}\!\left(#2\right)}
}
\newcommand{\opcat}[1]{{#1}^{\textup{op}}}
\newcommand{\revcat}[1]{{#1}^{\textup{rev}}}
\newcommand{\textcat}[1]{\textup{\textsf{#1}}}
\newcommand{\rmodcat}[2][]{\textcat{Mod}_{#1}\textcat{-}#2}
\newcommand{\lmodcat}[2][]{#2\textcat{-Mod}_{#1}}
\newcommand{\bimodcat}[3][]{#2\textcat{-Mod}_{#1}\textcat{-}#3}
\newcommand{\moreq}[1]{\overset{\textup{m.e.}}{#1}}

% Special notation.
\newcommand{\chr}{\textup{char}}
\newcommand{\Tr}{\textup{Tr}}
\newcommand{\trv}{\textup{tr}}
\newcommand{\Irr}{\textup{Irr}}
\newcommand{\Gr}{\textup{Gr}}
\newcommand{\dimh}[2]{\left(#1, #2\right)}
\newcommand{\Dim}{\textup{Dim}}
\newcommand{\FPdim}{\textup{FPdim}}
\newcommand{\BrPic}[1][]{
	\ifthenelse{\equal{#1}{}}
		{\textup{BrPic}}
		{\textup{BrPic}\!\left(#1\right)}
}
\newcommand{\BrPicC}[1][]{
	\ifthenelse{\equal{#1}{}}
		{\underline{\BrPic}}
		{\underline{\BrPic}\!\left(#1\right)}
}
\newcommand{\BrPicCC}[1][]{
	\ifthenelse{\equal{#1}{}}
		{\underline{\BrPicC}}
		{\underline{\BrPicC}\!\left(#1\right)}
}
\newcommand{\Aut}{\textup{Aut}}
\newcommand{\Inn}{\textup{Inn}}
\newcommand{\Out}{\textup{Out}}

% Hiragana "yo" for the Yoneda embeddings.
\newcommand{\yo}{\text{\usefont{U}{min}{m}{n}\symbol{'210}}}
\DeclareFontFamily{U}{min}{}
\DeclareFontShape{U}{min}{m}{n}{<-> udmj30}{}

% Representation theory notation.
\newcommand{\Sym}{\textup{Sym}}
\newcommand{\Alt}{\textup{Alt}}


%%%%%%%%%%%%
% PLOTTING %
%%%%%%%%%%%%

\usepackage{tikz-cd}
\usepackage{tikz}
\usepackage{spath3}
\usetikzlibrary{arrows.meta, decorations.markings, decorations.pathreplacing, knots}

% TikZ macro for inclusions and projections.
\def\fusioninclusion[#1]#2(#3, #4, #5, #6)#7(#8){
	\draw[#1] ({#3 + #5/2}, {#4}) arc(0:180:{#5/2} and {#6/2});
	\draw[#1] ({#3 - #5/2}, {#4}) to ({#3 - #5/2}, {#4 - #6/2}) to ({#3 + #5/2}, {#4 - #6/2}) to ({#3 + #5/2}, {#4});
	% Label the node.
	\node at ({#3}, {#4 - 1/8}) {#8};
}
\def\fusionprojection[#1]#2(#3, #4, #5, #6)#7(#8){
	\draw[#1] ({#3 - #5/2}, {#4}) arc(180:360:{#5/2} and {#6/2});
	\draw[#1] ({#3 - #5/2}, {#4}) to ({#3 - #5/2}, {#4 + #6/2}) to ({#3 + #5/2}, {#4 + #6/2}) to ({#3 + #5/2}, {#4});
	% Label the node.
	\node at ({#3}, {#4}) {#8};
}


%%%%%%%%%%%%
% DOCUMENT %
%%%%%%%%%%%%

\begin{document}

%\begin{titlingpage}
%\maketitle
%\end{titlingpage}
%\newpage

%\setcounter{page}{2}
%\tableofcontents
%\newpage

\thispagestyle{fancy}


\section{The \texorpdfstring{$(\mathbbm{Z}/2\mathbb{Z} \times \mathbb{Z}/2\mathbb{Z}, 4)$}{Z/2ZxZ/2Z} Near-Group}\label{chp:introduction}
\sectionbar{1}{1pt}{-2}{0}

\noindent Let $\mathcat{C}$ denote the (unitary) near-group fusion category of type $(G, n)$, where $G \coloneqq \mathbb{Z}/2\mathbb{Z} \times \mathbb{Z}/2\mathbb{Z}$ and $n = \abs{G}$. More precisely, we will denote $G = \{\mathbbm{1}, (g, 1), (1, g), (g, g)\}$ and $\Irr(\mathcat{C}) = G \sqcup \{X\}$. Note that $\FPdim(x) = 1$ for $x \in G$ and $\FPdim(X) = 2 + 2\sqrt{2}$. The aim of this document is to explain the current state of our attempt to classify the simple algebra objects in $\mathcat{C}$.
\newline\newline
\noindent It is easy to show (via the code or just checking manually) that the candidates for simple algebra objects in $\mathcat{C}$ are as follows:
\begin{enumerate}[start=1, leftmargin=1.5cm, label={(\arabic*).}]
\item $A_1^1 \coloneqq G \moreq{\sim} G \oplus X^{\oplus 4}$,
\item $A_1^2 \coloneqq \langle{(1, g)}\rangle \oplus X \moreq{\sim} G \oplus X^{\oplus 2}$,
\item $A_1^3 \coloneqq \langle{(g, 1)}\rangle \oplus X \moreq{\sim} G \oplus X^{\oplus 2}$,
\item $A_1^4 \coloneqq \langle{(g, g)}\rangle \oplus X \moreq{\sim} G \oplus X^{\oplus 2}$,
\item $A_1^5 \coloneqq \langle{(1, g)}\rangle \moreq{\sim} \langle{(1, g)}\rangle \oplus X^{\oplus 2}$,
\item $A_1^6 \coloneqq \langle{(1, g)}\rangle \moreq{\sim} \langle{(g, g)}\rangle \oplus X^{\oplus 2}$,
\item $A_1^7 \coloneqq \langle{(1, g)}\rangle \moreq{\sim} \langle{(g, 1)}\rangle \oplus X^{\oplus 2}$,
\item $A_1^8 \coloneqq \langle{(g, g)}\rangle \moreq{\sim} \langle{(1, g)}\rangle \oplus X^{\oplus 2}$,
\item $A_1^9 \coloneqq \langle{(g, g)}\rangle \moreq{\sim} \langle{(g, g)}\rangle \oplus X^{\oplus 2}$,
\item $A_1^{10} \coloneqq \langle{(g, g)}\rangle \moreq{\sim} \langle{(g, 1)}\rangle \oplus X^{\oplus 2}$,
\item $A_1^{11} \coloneqq \langle{(g, 1)}\rangle \moreq{\sim} \langle{(1, g)}\rangle \oplus X^{\oplus 2}$,
\item $A_1^{12} \coloneqq \langle{(g, 1)}\rangle \moreq{\sim} \langle{(g, g)}\rangle \oplus X^{\oplus 2}$,
\item $A_1^{13} \coloneqq \langle{(g, 1)}\rangle \moreq{\sim} \langle{(g, 1)}\rangle \oplus X^{\oplus 2}$,
\item $A_1^{14} \coloneqq \mathbbm{1} \moreq{\sim} G \oplus X^{\oplus 4}$,
\item $A_1^{15} \coloneqq G \moreq{\sim} \mathbbm{1} \oplus X$.
\end{enumerate}
\noindent Here, $\langle{(g, 1)}\rangle \coloneqq \mathbbm{1} \oplus (g, 1)$ for example, and we write $\moreq{\sim}$ to denote Morita equivalence of algebra objects. Note that all of these, except for $A_1^2$, $A_1^3$ and $A_1^4$, are Morita equivalent subgroups of $G$. The main question we are interested in is whether any of these three candidates admit a simple algebra structure, and if so, how many?
\newline\newline
\noindent I'll divide the rest of this document into three parts.
\begin{enumerate}[start=1, leftmargin=1.5cm, label={(\Roman*).}]
\item \hyperref[chp:lifts_from_vecg]{Which candidates lifting from $\textcat{Vec}_G$ exist?}
\item \hyperref[chp:dual_cats]{What are the dual fusion categories at play?}
\item \hyperref[chp:z2_plus_x]{Under what circumstances does $\mathbb{Z}/2\mathbb{Z} \oplus X$ admit a simple algebra structure?}
\end{enumerate}
\newpage

\section{Algebras Lifting from \texorpdfstring{$\textcat{Vec}_G$}{VecG}}\label{chp:lifts_from_vecg}
\sectionbar{1}{1pt}{-2}{0}

\noindent We first try to answer the question of which candidates actually exist when the algebra object is a subgroup of $G$. Thanks to the classification of simple algebra objects in $\textcat{Vec}_G$, we know that $\langle{(g, 1)}\rangle$, $\langle{(1, g)}\rangle$ and $\langle{(g, g)}\rangle$ admit a single algebra structure, while $G$ admits two. By direct computation, one can show that $\mathbbm{1} \oplus X$ admits only a single algebra structure, so both $A_1^1$ and $A_1^{15}$ must exist. While $A_1^{15}$ is self-dual (by which we mean their dual fusion categories are not equivalent to $\mathcat{C}$), $A_1^1$ has many candidates for its dual fusion category.
\newline\newline
\noindent Determining which of the algebra structures on the copies of $\mathbb{Z}/2\mathbb{Z}$ exist is a bit trickier. Using the code's \texttt{findDualRings()} function, we find that candidates $A_1^5$, $A_1^9$ and $A_1^{13}$ cannot be self-dual. The other candidates lifting from copies of $\mathbb{Z}/2\mathbb{Z}$ must be self-dual, however. Checking bimodule compatibility using the code reveals that we have two potential cases.
\begin{enumerate}[start=1, leftmargin=1.5cm, label={(\arabic*).}]
\item All three of $A_1^5$, $A_1^9$, $A_1^{13}$ exist.
\item Exactly two of $A_1^5$, $A_1^9$, $A_1^{13}$ exist, and the remaining $\mathbb{Z}/2\mathbb{Z}$ subgroup is self-dual.
\end{enumerate}
\noindent In both cases, the Morita equivalence class of $\mathcat{C}$ must contain at least one other fusion category.
\newline\newline
\noindent \textcolor{red}{Is there more we can say here? The Brauer-Picard group could probably help us figure out which case we're in.}
\newpage

\section{The Dual Fusion Categories}\label{chp:dual_cats}
\sectionbar{1}{1pt}{-2}{0}

\noindent The code's \texttt{findDualRings()} function finds exactly two potential dual fusion categories to $A_1^5$, $A_1^9$ and $A_1^{13}$, which are defined as follows:
\begin{gather*}
\Irr(\mathcat{D}_1) = G \sqcup G\{X\}, \\
(g, 1) \otimes X \cong X \otimes (g, 1), \quad (1, g) \otimes X \cong X \otimes (g, g), \quad (g, g) \otimes X \cong X \otimes (1, g), \\
X \otimes X \cong \mathbbm{1} \oplus X \oplus (g, 1)X;
\end{gather*}
\begin{gather*}
\Irr(\mathcat{D}_1') = G \sqcup G\{X\}, \\
(g, 1) \otimes X \cong X \otimes (g, 1), \quad (1, g) \otimes X \cong X \otimes (g, g), \quad (g, g) \otimes X \cong X \otimes (1, g), \\
X \otimes X \cong \mathbbm{1} \oplus (1, g)X \oplus (g, g)X.
\end{gather*}
\noindent Here, $\FPdim(X) = 1 + \sqrt{2}$. Note that the only difference in these rules is $X \otimes X$. We suspect that both of these are $\mathbb{Z}/2\mathbb{Z}$-graded extensions of the $\mathbb{Z}/2\mathbb{Z}$ Haagerup-Izumi fusion category, with $\mathcat{D}_1$ being quasi-trivial.
\newline\newline
\noindent Both $\mathcat{D}_1$ and $\mathcat{D}_1'$ have 6 simple algebra objects lifting from $\textcat{Vec}_G$. This completely classifies the simple algebra objects in $\mathcat{D}_1'$. For $\mathcat{D}_1$, however, there is an additional unique algebra structure on $\mathbbm{1} \oplus X$ lifting from the $\mathbb{Z}/2\mathbb{Z}$ Haagerup-Izumi subcategory $\langle{\mathbbm{1}, (g, 1), X}\rangle$. The algebra object $\mathbbm{1} \oplus X$ in $\mathcat{D}_1'$ has a unique candidate for its dual fusion category, which coincides with the unique dual fusion category candidate for $\mathbb{Z}/2\mathbb{Z} \oplus X$ in $\mathcat{C}$. This category is defined as follows:
\begin{gather*}
\Irr(\mathcat{D}_2) = \mathbb{Z}/2\mathbb{Z} \sqcup \{X\} \sqcup (\mathbb{Z}/2\mathbb{Z})\{Y\} \sqcup \{Z\}, \\
g \otimes X \cong X \cong X \otimes g, \quad X \otimes X \cong \mathbbm{1} \oplus g, \\
g \otimes Y \cong gY \cong Y \otimes g, \quad Y \otimes Y \cong \mathbbm{1} \oplus Y \oplus gY, \\
g \otimes Z \cong Z \cong Z \otimes g, \quad Z \otimes Z \cong \mathbbm{1} \oplus g \oplus Y^{\oplus 2} \oplus gY^{\oplus 2}, \\
X \otimes Y \cong X \otimes gY \cong Y \otimes X \cong gY \otimes X \cong Z, \\
X \otimes Z \cong Z \otimes X \cong Y \oplus gY, \\
Y \otimes Z \cong Z \otimes Y \cong X \oplus Z^{\oplus 2}.
\end{gather*}
\noindent We have denoted $\mathbb{Z}/2\mathbb{Z} = \{\mathbbm{1}, g\}$ here. Because this is a bit messy, we clarify that $\langle{\mathbbm{1}, g, X}\rangle$ is a Tambara-Yamagami fusion category, whereas $\langle{\mathbbm{1}, g, Y, gY}\rangle$ is a $\mathbb{Z}/2\mathbb{Z}$ Haagerup-Izumi fusion category. We also remark that $\FPdim(X) = \sqrt{2}$, $\FPdim(Y) = 1 + \sqrt{2}$ and $\FPdim(Z) = 2 + \sqrt{2}$.
\newpage
\noindent The simple algebra object candidates in $\mathcat{D}_2$ are a little tricky. It's perhaps worth mentioning what the simple algebra object candidates in these dual fusion categories are. The case of $\mathcat{D}_1'$ is easy: as previously mentioned, every candidate lifts from $\textcat{Vec}_G$. In particular, since the number of simple algebra objects is an invariant of Morita equivalence, this tells us that $\mathbb{Z}/2\mathbb{Z} \oplus X$ cannot have a simple algebra structure in $\mathcat{C}$ if $\mathcat{C} \moreq{\cong} \mathcat{D}_1'$. Conversely, if $\mathcat{C} \moreq{\cong} \mathcat{D}_1$, then we have 7 simple algebra objects \textcolor{red}{since $\mathbbm{1} \oplus X$ lifts uniquely in $\mathcat{D}_1$ (check)}, so there must be exactly one simple algebra structure on one of the copies of $\mathbb{Z}/2\mathbb{Z} \oplus X$ in $\mathcat{C}$.
\newline\newline
\noindent In $\mathcat{D}_2$, the simple algebra object candidates are less obvious:
\begin{enumerate}[start=1, leftmargin=1.5cm, label={(\arabic*).}]
\item $A_2^1 \coloneqq \mathbbm{1} \oplus Y \oplus Z \moreq{\sim} \mathbbm{1} \oplus g \oplus Y \oplus gY \oplus Z^{\oplus 2}$,
\item $A_2^2 \coloneqq \mathbbm{1} \oplus gY \oplus Z \moreq{\sim} \mathbbm{1} \oplus g \oplus Y \oplus gY \oplus Z^{\oplus 2}$,
\item $A_2^3 \coloneqq \mathbbm{1} \moreq{\sim} \mathbbm{1} \oplus g \moreq{\sim} \mathbbm{1} \oplus Y \oplus gY \moreq{\sim} \mathbbm{1} \oplus g \oplus Y^{\oplus 2} \oplus gY^{\oplus 2}$,
\item $A_2^4 \coloneqq \mathbbm{1} \oplus gY \moreq{\sim} \mathbbm{1} \oplus g \oplus Y \oplus gY$,
\item $A_2^5 \coloneqq \mathbbm{1} \oplus gY \moreq{\sim} \mathbbm{1} \oplus Y \moreq{\sim} \mathbbm{1} \oplus g \oplus Y \oplus gY$,
\item $A_2^6 \coloneqq \mathbbm{1} \oplus Y \moreq{\sim} \mathbbm{1} \oplus g \oplus Y \oplus gY$.
\end{enumerate}
\noindent While most of these lift from the Haagerup-Izumi subcategory, $A_2^1$ and $A_2^2$ are exceptions. \textcolor{red}{Mention which of these give Morita equivalences to $\mathcat{C}$ and which give Morita equivalences to $\mathcat{D}_1$.}
%Rep 1: 3458
%	Near-group
%	Lots of funky options (rank 8, so likely D_1)
%
%Rep 2: 31528
%	Near-group
%
%Rep 3: 31885
%	Near-group (exactly two copies of this algebra)
%
%Rep 4: 32242
%	Lots of funky options (rank 8, so likely D_1)
%
%Rep 5: 34749
%	Self-dual (does this lift uniquely from the near-group?)
%
%Rep 6: 35303
%	Took too long to finish (there are only bimodules with D_1, though)
\newpage

\section{When Is \texorpdfstring{$\mathbb{Z}/2\mathbb{Z} \oplus X$}{Z/2Z + X} an Algebra Object?}\label{chp:z2_plus_x}
\sectionbar{1}{1pt}{-2}{0}

\noindent Recall that $\mathcat{C}$ is Morita equivalent either to $\mathcat{D}_1$ or $\mathcat{D}_1'$, where the former has 7 simple algebra objects and the latter has 6. Since the number of simple algebra objects is an invariant of Morita equivalence, we can state the following.
\begin{enumerate}[start=1, leftmargin=1.5cm, label={(\arabic*).}]
\item There is a (unique) simple algebra structure on $\mathbb{Z}/2\mathbb{Z} \oplus X$ in $\mathcat{C}$ if and only if $\mathcat{C} \moreq{\cong} \mathcat{D}_1$.
\end{enumerate}
\noindent Recall that $\mathbb{Z}/2\mathbb{Z} \oplus X \moreq{\sim} G \oplus X^{\oplus 2}$. If we assume that $G \oplus X^{\oplus 2}$ contains $G$ as a subalgebra object, then the factoring argument tells us that the dual fusion category of $G$ contains a simple algebra object with Frobenius-Perron dimension $\FPdim(G \oplus X^{\oplus 2})/\FPdim(G) = 2 + \sqrt{2}$. Note that $\mathcat{C}$ has no object with this dimension, and $G \moreq{\sim} \mathbbm{1} \oplus X$ is self-dual. Thus, we obtain another condition on the existence of an algebra structure on $\mathbb{Z}/2\mathbb{Z} \oplus X$.
\begin{enumerate}[start=2, leftmargin=1.5cm, label={(\arabic*).}]
\item If there is a simple algebra structure on $\mathbb{Z}/2\mathbb{Z} \oplus X$ and it contains $G$ as a subalgebra object, then this subalgebra object is $A_1^1 \moreq{\sim} G \oplus X^{\oplus 4}$, and is not self-dual.
\end{enumerate}
\noindent The code finds several candidates for the dual fusion category of $A_1^1$, including both $\mathcat{C}$ and $\mathcat{D}_1$. Some of these fusion categories (like $\mathcat{D}_1$, as well as some others) admit an algebra object candidate of Frobenius-Perron dimension $2 + \sqrt{2}$.

\end{document}