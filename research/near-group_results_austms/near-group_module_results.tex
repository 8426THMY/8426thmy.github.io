%%%%%%%%%%%%%%%%%%%%%%%%%%%%%%%%%%%%%%%%
%                                      %
% Random LaTeX Beamer Template         %
%                                      %
% Created by Thomas Dunmore.           %
% Last updated 2024/11/07.             %
%                                      %
%%%%%%%%%%%%%%%%%%%%%%%%%%%%%%%%%%%%%%%%

%%%%%%%%%
% SETUP %
%%%%%%%%%

\documentclass{beamer}
\mode<presentation> {
	\usetheme{Madrid}
}
\setbeamertemplate{frametitle continuation}[from second][]

\usepackage{csquotes}
\usepackage{graphicx}
\usepackage{booktabs}

% Title.

\title[Algebra Objects in Near-Groups]{Algebra Objects in Near-Group Fusion Categories}
\author[Thomas Dunmore]{
	Thomas Dunmore \\
	{\footnotesize Supervisor: Assoc. Prof. Pinhas Grossman} \\
	{\footnotesize Co-supervisor: Dr. Anna Romanov}
}
\institute[UNSW]{
	University of New South Wales \\
	\medskip
	\textit{t.dunmore@unsw.edu.au}
}
\date{2025/12/09}
\titlegraphic{\includegraphics[width=2cm]{unsw-crest}}
\logo{\includegraphics[width=3.5cm]{qr-code}}

% Bibliography.
\usepackage[%
	backend = biber,
	% BibLaTeX-Math package: https://github.com/konn/biblatex-math
	style = math-alphabetic,
	giveninits = true,
	dashed = false,
	url = false,
	doi = false,
	sorting = anyvt,
	minalphanames = 3,
	maxalphanames = 4,
	maxcitenames = 5,
	maxbibnames = 5
]{biblatex}
% Use the default font size for the bibliography.
\renewcommand*{\bibfont}{\normalsize}
% Use title case rather than sentence case for references.
\DeclareFieldFormat{titlecase}{#1}
% Put last names before first names.
\DeclareNameAlias{default}{family-given}
% Used for articles with appendices written by other authors.
\NewBibliographyString{bywithappendix}
\DefineBibliographyStrings{english}{
	bywithappendix = {with an appendix by}
}
% Specify the bibliography data file to use.
\addbibresource{references.bib}
% Automatically add every reference to the bibliography.
\nocite{*}

% Plotting.
\usepackage{tikz-cd}
\usepackage{spath3}
\usetikzlibrary{arrows.meta, decorations.markings, decorations.pathreplacing, bending, knots}


%%%%%%%%%%%%%%%%
% ENVIRONMENTS %
%%%%%%%%%%%%%%%%

% Theorem environments.
\theoremstyle{plain}
\newtheorem{proposition}[theorem]{Proposition}
\newtheorem{conjecture}[theorem]{Conjecture}

\theoremstyle{definition}
\newtheorem{question}[theorem]{Question}
\newtheorem{remark}[theorem]{Remark}
\newtheorem{notation}[theorem]{Notation}


%%%%%%%%%%%%
% NOTATION %
%%%%%%%%%%%%

% Adds integral notation like \oiint.
\usepackage{esint}
% Blackboard bold symbols.
\usepackage{bbm}

\usepackage{accents}
% Tilde notation for vectors.
\newcommand{\ut}[1]{\underaccent{\tilde}{#1}}
% Arrow notation for vectors.
\usepackage{harpoon}
% Dirac bra-ket notation for quantum states.
\usepackage{braket}

% Differential formatting.
\usepackage{ifthen}
\usepackage{etoolbox}
\newcommand*{\ndiff}[1]{\mathrm{d}#1}
\newcommand*{\sdiff}[1]{\mathop{}\!\ndiff{#1}}
\newcommand{\rdiff}[3][]{
	\ifthenelse{\equal{#1}{}}
		{\frac{\mathrm{d}#2}{\mathrm{d}#3}}
		{\frac{\mathrm{d}^{#1}#2}{\forcsvlist\ndiff{#3}}}
}
\newcommand*{\npiff}[1]{\mathrm{\partial}#1}
\newcommand*{\spiff}[1]{\mathop{}\!\npiff{#1}}
\newcommand{\rpiff}[3][]{
	\ifthenelse{\equal{#1}{}}
		{\frac{\mathrm{\partial}#2}{\mathrm{\partial}#3}}
		{\frac{\mathrm{\partial}^{#1}#2}{\forcsvlist\npiff{#3}}}
}
% Inexact differential for physics.
\newcommand*{\dbar}[1]{\mathop{}\!\mathrm{\dj}#1}

% Metrics, inner products and norms.
\usepackage{mathtools}
\DeclarePairedDelimiter{\abs}{\lvert}{\rvert}
\DeclarePairedDelimiter{\inprod}{\langle}{\rangle}
\DeclarePairedDelimiter{\norm}{\lVert}{\rVert}
% This is used if we want an empty norm. 
\newcommand{\blank}{{}\cdot{}}

% Function notation.
\newcommand{\id}{\textup{id}}
\newcommand{\coker}{\textup{coker}}
\newcommand{\im}{\textup{im}}
\newcommand{\ev}{\textup{ev}}
\newcommand{\coev}{\textup{coev}}

% Category font.
\newcommand{\mathcat}[1]{\mathcal{#1}}

% Category theory notation.
\newcommand{\Ob}{\textup{Ob}}
\newcommand{\Hom}[2][]{
	\ifthenelse{\equal{#2}{}}
		{\textup{Hom}_{#1}}
		{\textup{Hom}_{#1}\!\left(#2\right)}
}
\newcommand{\IntHom}[2][]{
	\ifthenelse{\equal{#2}{}}
		{\underline{\textup{Hom}}_{#1}}
		{\underline{\textup{Hom}}_{#1}\!\left(#2\right)}
}
\newcommand{\End}[2][]{
	\ifthenelse{\equal{#2}{}}
		{\textup{End}_{#1}}
		{\textup{End}_{#1}\!\left(#2\right)}
}
\newcommand{\IntEnd}[2][]{
	\ifthenelse{\equal{#2}{}}
		{\underline{\textup{End}}_{#1}}
		{\underline{\textup{End}}_{#1}\!\left(#2\right)}
}
\newcommand{\Fun}[2][]{
	\ifthenelse{\equal{#2}{}}
		{\textcat{Fun}_{#1}}
		{\textcat{Fun}_{#1}\!\left(#2\right)}
}
\newcommand{\opcat}[1]{{#1}^{\textup{op}}}
\newcommand{\revcat}[1]{{#1}^{\textup{rev}}}
\newcommand{\textcat}[1]{\textup{\textsf{#1}}}
\newcommand{\rmodcat}[2][]{\textcat{Mod}_{#1}\textcat{-}#2}
\newcommand{\lmodcat}[2][]{#2\textcat{-Mod}_{#1}}
\newcommand{\bimodcat}[3][]{#2\textcat{-Mod}_{#1}\textcat{-}#3}
\newcommand{\moreq}[1]{\overset{\textup{m.e.}}{#1}}

% Special notation.
\newcommand{\chr}{\textup{char}}
\newcommand{\Tr}{\textup{Tr}}
\newcommand{\trv}{\textup{tr}}
\newcommand{\Irr}{\textup{Irr}}
\newcommand{\Gr}{\textup{Gr}}
\newcommand{\dimh}[2]{\left(#1, #2\right)}
\newcommand{\Dim}{\textup{Dim}}
\newcommand{\FPdim}{\textup{FPdim}}
\newcommand{\BrPic}[1][]{
	\ifthenelse{\equal{#1}{}}
		{\textup{BrPic}}
		{\textup{BrPic}\!\left(#1\right)}
}
\newcommand{\BrPicC}[1][]{
	\ifthenelse{\equal{#1}{}}
		{\underline{\BrPic}}
		{\underline{\BrPic}\!\left(#1\right)}
}
\newcommand{\BrPicCC}[1][]{
	\ifthenelse{\equal{#1}{}}
		{\underline{\BrPicC}}
		{\underline{\BrPicC}\!\left(#1\right)}
}
\newcommand{\Aut}{\textup{Aut}}
\newcommand{\Inn}{\textup{Inn}}
\newcommand{\Out}{\textup{Out}}

% Hiragana "yo" for the Yoneda embeddings.
\newcommand{\yo}{\text{\usefont{U}{min}{m}{n}\symbol{'210}}}
\DeclareFontFamily{U}{min}{}
\DeclareFontShape{U}{min}{m}{n}{<-> udmj30}{}

% Representation theory notation.
\newcommand{\Sym}{\textup{Sym}}
\newcommand{\Alt}{\textup{Alt}}


%%%%%%%%%%%%
% DOCUMENT %
%%%%%%%%%%%%

\begin{document}

%%%%%%%%%%%%
% Overview %
%%%%%%%%%%%%

\begin{frame}
\titlepage
\end{frame}

\logo{}

\begin{frame}
\frametitle{Overview}
\begin{center}
\begin{minipage}{\widthof{(3) Modules Over Near-Groups}}
\setlength{\parskip}{4ex}
\tableofcontents
\end{minipage}
\end{center}
\end{frame}

%%%%%%%%%%%%%%%%%%%%%
% Fusion Categories %
%%%%%%%%%%%%%%%%%%%%%

\section{Fusion Categories}

\begin{frame}
\centerline{\Huge\textcolor{structure}{\underline{Fusion Categories}}}
\end{frame}

\begin{frame}
\frametitle{What Is a Fusion Category?}
I like to think of fusion categories as generalizations of the category $\textcat{Rep}(G)$ of finite-dimensional representations of a finite group $G$ over $\mathbb{C}$. To see how this works, let's see what kind of properties $\textcat{Rep}(G)$ has.
\medskip
\begin{itemize}
	\item \textcolor{structure}{Linear}: hom-sets are vector spaces.
	\item \textcolor{structure}{Semisimple}: every representation is a direct sum of irreducibles.
	\item \textcolor{structure}{Finite}: there are only finitely many irreducibles.
	\item \textcolor{structure}{Simple unit}: the trivial representation, $\mathbbm{1}$, is irreducible.
	\item \textcolor{structure}{Monoidal}: we have (linear) tensor products with unit $\mathbbm{1}$.
	\item \textcolor{structure}{Rigid}: every representation $X$ has a contragredient dual $X^{*}$.
\end{itemize}
\begin{definition}[Fusion Category]
A category with these adjectives is called a \textcolor{structure}{fusion category}.
\end{definition}
\end{frame}

\begin{frame}
\frametitle{Classical Examples of Fusion Categories}
\begin{example}[Graded Vector Spaces]
The category \textcolor{structure}{$\textcat{Vec}_G$} of finite-dimensional vector spaces (over $\mathbb{C}$, say) graded by a finite group $G$ is a fusion category. This category has a simple object for each $g \in G$, monoidal product $g \otimes h \cong gh$ and duals $g^{*} = g^{-1}$. One obtains distinct fusion categories by ``twisting'' $\otimes$ by \textcolor{structure}{3-cocycles} of $G$.
\end{example}
\begin{example}[Quantum $\mathfrak{sl}_2$]
Given a Lusztig quantum group $U_q^L(\mathfrak{g})$ for some finite-dimensional simple Lie algebra $\mathfrak{g}$ and $2\ell$-th root of unity $q$, one builds a fusion category by taking a certain quotient of its category of representations. For $\mathfrak{g} = \mathfrak{sl}_2$, we get simple objects $\{W_i\}_{i=0}^{\ell-2}$ satisfying the \textcolor{structure}{truncated Clebsch-Gordon rules},
\begin{align*}
\begin{split}
W_i \otimes W_j \cong \bigoplus\nolimits_{k=\max\{i+j-\ell+2, 0\}}^{\min\{i, j\}}{W_{i+j-2k}}.
\end{split}
\end{align*}
\end{example}
\end{frame}

\begin{frame}
\frametitle{Near-Group Fusion Categories}
In general, new fusion categories are very hard to find! Almost all known examples come from those shown above via some kind of construction. One notable class of exceptions is the \textcolor{structure}{near-group fusion categories}.
\begin{definition}[Near-Group Fusion Categories]
A fusion category with only one non-invertible object $X$ (so $X^{*} \otimes X \ncong \mathbbm{1}$) is called \textcolor{structure}{near-group}. Then $X \otimes X \cong \bigoplus_{g \in G}{g} \oplus X^{\oplus m}$, where $G$ is its group of invertibles and $m \in \mathbb{Z}_{\geq 0}$. We say such a near-group has \textcolor{structure}{type $(G, m)$}.
\end{definition}
We are most interested in the case where $G$ is Abelian and $m \in \mathbb{Z}_{>0}\abs{G}$. The other cases have been classified (at least in the unitary setting) by Tambara-Yamagami, Izumi and Evans-Gannon.
\begin{example}[First Examples of Near-Groups]
The category $\textcat{Rep}(S_3)$ is near-group of type $(\mathbb{Z}/2\mathbb{Z}, 1)$, while the Fibonnaci and Yang-Lee fusion categories are near-group of type $(\{\mathbbm{1}\}, 1)$.
\end{example}
\end{frame}

%%%%%%%%%%%%%%%%%%%%%
% Module Categories %
%%%%%%%%%%%%%%%%%%%%%

\section{Module Categories}

\begin{frame}
\centerline{\Huge\textcolor{structure}{\underline{Module Categories}}}
\end{frame}

\begin{frame}
\frametitle{What Is a Module Category?}
All fusion categories define a ring with addition $\oplus$ and multiplication $\otimes$. One can similarly categorify the notion of modules over these rings.
\begin{definition}[Fusion Module Category]
A \textcolor{structure}{module category} over a fusion category $\mathcat{C}$ is a linear, semisimple, finite, category $\mathcat{M}$ with a linear monoidal functor $\mathcat{C} \to \End{\mathcat{M}}$. If $\mathcat{M}$ is not a direct sum of non-trivial module categories, I will call it \textcolor{structure}{fusion}.
\end{definition}
This is nice, but why bother? Well, there are several reasons...
\medskip
\begin{itemize}
	\item Knowing which modules can and can't exist can place additional combinatorial constraints on the existence of a fusion category.
	\item The \textcolor{structure}{dual category $\mathcat{C}_{\mathcat{M}}^{*} \coloneqq \Fun[\mathcat{C}]{\mathcat{M}, \mathcat{M}}$} of $\mathcat{C}$-module endofunctors of a fusion module category $\mathcat{M}$ is always a fusion category.
\end{itemize}
\medskip
Unfortunately, finding new module categories from first principles is often just as hard as finding new fusion categories. Luckily, there's another way.
\end{frame}

\begin{frame}
\frametitle{Algebra Objects and Ostrik's Theorem}
\begin{definition}[Algebra and Module Objects]
An \textcolor{structure}{algebra object} is an object $A$ with morphisms $m : A \otimes A \to A$ and $u : \mathbbm{1} \to A$ satisfying certain axioms. A \textcolor{structure}{module object} over $A$ is an object $M$ with an action $a : M \otimes A \to M$ compatible with $m$ and $u$.
\end{definition}
\begin{theorem}[\citeauthor{Ostrik_2003}, \citeyear{Ostrik_2003}]
The category $\rmodcat[\mathcat{C}]{A}$ of module objects over a ``simple'' algebra object $A \in \Ob(\mathcat{C})$ is a fusion module category. Conversely, if $M$ is a simple object in a fusion module category, then $\IntHom{M, M}$ is a simple algebra object.
\end{theorem}
Usually, it is significantly easier to come up with potential candidates for algebra objects. In most cases, one can also figure out what the module category should look like.
\end{frame}

\begin{frame}
\frametitle{Examples of Module Categories}
\begin{example}[Algebras and Modules of $\textcat{Vec}_G$]
An algebra object in $\textcat{Vec}_G$ is a \textcolor{structure}{$G$-graded} associative, unital algebra, and the simple ones are parametrized by pairs $(H, \psi)$ of a subgroup $H \leq G$ and a 2-cocycle $\psi$ of $H$. The actual object is just the group algebra $\bigoplus_{h \in H}{h}$ with multiplication ``twisted'' by $\psi$, while the corresponding module category looks like $\textcat{Vec}_{G/H}$ with the usual $G$-action twisted by $\psi$.
\end{example}
\begin{example}[Dual Fusion Categories of $\textcat{Vec}_G$]
In any fusion category, the unit $\mathbbm{1}$ is an algebra object with dual fusion category $\mathcat{C}$. In $\textcat{Vec}_G$, the algebra $G$ has dual fusion category $\textcat{Rep}(G)$.%
\newline\newline
More interestingly, in $\textcat{Vec}_{\mathbb{Z}/4\mathbb{Z}}$, the algebra object $\mathbb{Z}/2\mathbb{Z}$ has dual fusion category $\textcat{Vec}_{\mathbb{Z}/2\mathbb{Z} \times \mathbb{Z}/2\mathbb{Z}}$ with non-trivially twisted associativity.
\end{example}
\end{frame}

%%%%%%%%%%%%%%%%%%%%%%%%%%%%
% Modules Over Near-Groups %
%%%%%%%%%%%%%%%%%%%%%%%%%%%%

\section{Modules Over Near-Groups}

\begin{frame}
\centerline{\Huge\textcolor{structure}{\underline{Modules Over Near-Groups}}}
\end{frame}

\begin{frame}
\frametitle{Setting the Stage}
Let $\mathcat{C}$ be a near-group of type $(G, m\abs{G})$ for some finite Abelian group $G$ and $m \in \mathbb{Z}_{>0}$. To help classify its fusion module categories, we can make some observations.
\medskip
\begin{itemize}
	\item Any near-group contains (untwisted!) $\textcat{Vec}_G$ as a fusion subcategory.
	\item Any fusion module category over $\mathcat{C}$ restricts to a (not necessarily indecomposable) $\textcat{Vec}_G$-module category.
	\item Any simple algebra object in $\mathcat{C}$ restricts to one in $\textcat{Vec}_G$, and simple algebra objects in $\textcat{Vec}_G$ lift to ones in $\mathcat{C}$.
\end{itemize}
\medskip
The upshot is that our classification of simple algebra objects in $\textcat{Vec}_G$ tells us that every simple algebra object in $\mathcat{C}$ is $H \oplus X^{\oplus a}$ for some $a \in \mathbb{Z}_{\geq 0}$, and every fusion $\mathcat{C}$-module category is a sum of fusion $\textcat{Vec}_G$-module categories.
\end{frame}

\begin{frame}
\frametitle{Counting Orbits}
Since $\mathcat{C}$ has only one non-invertible object, a fusion $\mathcat{C}$-module category can have at most two ``$G$-orbits''. Looking at dimensions reveals we always have exactly two, so every fusion $\mathcat{C}$-module category gives us a pair of ``Morita equivalent'' algebra objects via the internal end. These objects are related as follows.
\begin{proposition}
The category of modules over $H \oplus X^{\oplus a}$ has exactly two $G$-orbits, with the second corresponding to a simple algebra object $H' \oplus X^{\oplus a'}$ satisfying
\begin{enumerate}
	\item $\frac{a}{\abs{H}} + \frac{a'}{\abs{H'}} = m$;
	\item $\sqrt{\frac{\abs{H}\abs{H'}}{\abs{G}} + aa'} \in \mathbb{Z}_{>0}$.
\end{enumerate}
\end{proposition}
On the decategorified level, this is the end of the story: this completely classifies all so-called ``NIM-reps'' of the underlying fusion ring.
\end{frame}

\begin{frame}
\frametitle{Annoying Examples}
Unfortunately, the picture is more complicated upstairs. Here are some examples to illustrate the sort of issues one can encounter.
\begin{example}
Let $G = \mathbb{Z}/4\mathbb{Z}$ and $m = 1$. Then $(G, \mathbbm{1} \oplus X)$ corresponds to a unique module category, but $(G, G \oplus X^{\oplus 4})$ and $(\mathbb{Z}/2\mathbb{Z} \oplus X, G \oplus X^{\oplus 2})$ do not.
\end{example}
Things are slightly different if we consider the other group of order 4.
\begin{example}
Let $G = \mathbb{Z}/2\mathbb{Z} \times \mathbb{Z}/2\mathbb{Z}$ and $m = 1$. Then $(G, \mathbbm{1} \oplus X)$ and $(G, G \oplus X^{\oplus 4})$ both correspond to unique module categories, coming from the two non-cohomologous 2-cocycles on $G$.
\end{example}
What about $\mathbb{Z}/2\mathbb{Z} \oplus X$ when $G = \mathbb{Z}/2\mathbb{Z} \times \mathbb{Z}/2\mathbb{Z}$? Unlike before, we can't immediately rule it out precisely because $G \oplus X^{\oplus 4}$ is an algebra object.
\end{frame}

\begin{frame}
\frametitle{A Small Classification}
Luckily, these problem cases can be ruled out purely combinatorially when we assume $\abs{G}$ is square-free. In this case, we can say the following.
\begin{proposition}
Let $m = 1$ and $G$ be a finite cyclic group whose order is square-free. Then the fusion module categories of $\mathcat{C}$ correspond bijectively to those in $\textcat{Vec}_G$, with the subgroup $H$ corresponding to the pair $(H, \abs{G/H} \oplus X^{\oplus \abs{G/H}})$.
\end{proposition}
We've already seen that not all algebra objects follow this particular pattern. However, we have yet to find an example of an algebra object that isn't Morita equivalent to a subgroup of $G$.
\begin{conjecture}
For all $G$ and $m \in \mathbb{Z}_{>0}\abs{G}$, the fusion module categories of $\mathcat{C}$ correspond bijectively to fusion module categories of $\textcat{Vec}_G$.
\end{conjecture}
\end{frame}

%%%%%%%%%%%%%%
% References %
%%%%%%%%%%%%%%

\section*{References}

\begin{frame}[allowframebreaks]
\frametitle{References}
\footnotesize{
\printbibliography[heading = none]
}
\end{frame}

%%%%%%%%%%%%%
% Thank you %
%%%%%%%%%%%%%

\begin{frame}
\centerline{\Huge\textcolor{structure}{\underline{Thank you for listening!}}}
\end{frame}

\end{document} 